\startcomponent component_modele
\product product_specfonc
\startpart[title={Principes généraux du modèle conceptuel e-ReColNat augmenté}]

%%%%%%%%%%%%%%%%%%%%%%%%%%%%%%%%%%%%%%%%%%%%%%%%%%%%%%%%%%%%%%%%%%%%%%%%%%%%%%%%
%%%%%%%%%%%%%%%%%%%%%%%%%%%%%%%%%%%%%%%%%%%%%%%%%%%%%%%%%%%%%%%%%%%%%%%%%%%%%%%%
%%%%%%%%%%%%%%%%%%%%%%%%%%%%%%%%%%%%%%%%%%%%%%%%%%%%%%%%%%%%%%%%%%%%%%%%%%%%%%%%
%%%%%%%%%%%%%%%%%%%%%%%%%%%%%%%%%%%%%%%%%%%%%%%%%%%%%%%%%%%%%%%%%%%%%%%%%%%%%%%%
%%%%%%%%%%%%%%%%%%%%%%%%%%%%%%%%%%%%%%%%%%%%%%%%%%%%%%%%%%%%%%%%%%%%%%%%%%%%%%%%
\startchapter[title={Introduction}]

Cette partie présente les fondements et les entités de base du modèle conceptuel e-ReColNat augmenté afin de préciser dans quel cadre logique nous pensons les rapports entre les modules dont le développement nous incombe et les autres modules du portail coopératif.
Les informations sur le modèle exposées dans ce document sont donc partielles, et insistent sur l'esprit d'ensemble plus que sur le détail des diverses classes de sous entités métier.

\stopchapter
%%%%%%%%%%%%%%%%%%%%%%%%%%%%%%%%%%%%%%%%%%%%%%%%%%%%%%%%%%%%%%%%%%%%%%%%%%%%%%%%
%%%%%%%%%%%%%%%%%%%%%%%%%%%%%%%%%%%%%%%%%%%%%%%%%%%%%%%%%%%%%%%%%%%%%%%%%%%%%%%%
%%%%%%%%%%%%%%%%%%%%%%%%%%%%%%%%%%%%%%%%%%%%%%%%%%%%%%%%%%%%%%%%%%%%%%%%%%%%%%%%
%%%%%%%%%%%%%%%%%%%%%%%%%%%%%%%%%%%%%%%%%%%%%%%%%%%%%%%%%%%%%%%%%%%%%%%%%%%%%%%%
%%%%%%%%%%%%%%%%%%%%%%%%%%%%%%%%%%%%%%%%%%%%%%%%%%%%%%%%%%%%%%%%%%%%%%%%%%%%%%%%
\startchapter[title={Terminologie}]

%%%%%%%%%%%%%%%%%%%%%%%%%%%%%%%%%%%%%%%%%%%%%%%%%%%%%%%%%%%%%%%%%%%%%%%%%%%%%%%%
%%%%%%%%%%%%%%%%%%%%%%%%%%%%%%%%%%%%%%%%%%%%%%%%%%%%%%%%%%%%%%%%%%%%%%%%%%%%%%%%
%%%%%%%%%%%%%%%%%%%%%%%%%%%%%%%%%%%%%%%%%%%%%%%%%%%%%%%%%%%%%%%%%%%%%%%%%%%%%%%%
\startsection[title={Le modèle « e-RecolNat+ »}]

Le Modèle conceptuel « e-ReColNat augmenté » --- dont le nom est initialement apparu sous les formes « modèle e-ReColNat+ » ou « modèle augmenté », confirmées au fil de plusieurs réunions avec Julien Husson --- inventorie et organise l'ensemble des concepts, propriétés et relations sur lesquels reposent les modules développés par le laboratoire DICEN/WP5 (collaboratoire, visite virtuelle, portail, et toutes les interfaces de contribution et de navigation documentaires et sociales qui les constituent ou qu'ils supposent).
Le prédicat « augmenté » dénote que ce modèle conceptuel a pour enjeu de fournir des strates d'enrichissements supplémentaires aux entités botaniques identifiées dans le modèle de données du WP2 --- nourrie notamment par l'IPT du GBIF ---, dans le modèle métier des Herbonautes v2, et plus généralement, dans les modèles métier de l'ensemble des applications liées à e-ReColNat, les « augmentations » afférentes pouvant être d'ordre documentaire et social.
Il contient toutefois des entités qui lui sont propres ; pour exemples, les tags, les discussions, ou encore les entités convoquées dans les visites virtuelles, chacun de ces concepts étant lié à une ou plusieurs entités plus directement botaniques issues des modèles métier des applications du Muséum et des autres partenaires.
Ce modèle procède, d'une part, de l'analyse des pratiques et discours de botanistes professionnels et amateurs\footnote{Menée par Lisa Chupin.}, et d'autre part, d'une analyse plus globale des gestes savants portant sur des documents multimédias hyperliés en contexte de travail coopératif mettant en jeu les compétences scientifiques du laboratoire DICEN.

\stopsection
%%%%%%%%%%%%%%%%%%%%%%%%%%%%%%%%%%%%%%%%%%%%%%%%%%%%%%%%%%%%%%%%%%%%%%%%%%%%%%%%
%%%%%%%%%%%%%%%%%%%%%%%%%%%%%%%%%%%%%%%%%%%%%%%%%%%%%%%%%%%%%%%%%%%%%%%%%%%%%%%%
%%%%%%%%%%%%%%%%%%%%%%%%%%%%%%%%%%%%%%%%%%%%%%%%%%%%%%%%%%%%%%%%%%%%%%%%%%%%%%%%
\startsection[title={La base « e-ReColNat+ »},reference=terme:base]

La Base de données « e-ReColNat augmentée » réalise la persistance des informations identifiées par le modèle e-ReColNat+ (les informations constituant leur identité, leurs propriétés intrinsèques, les données qui les définissent et leurs sont directement rattachées, ainsi que les relations qu'elles entretiennent les unes avec les autres).
Différentes APIs (d'extraction et d'écriture) seront proposées pour rendre possible l'exploitation de certaines fonctions de cette base au sein de l'ensemble des modules du portail\footnote{Et peut-être au-delà de ce périmètre, si cela s'avère pertinent.}.
Voir \in[part:api].

\stopsection
\stopchapter
%%%%%%%%%%%%%%%%%%%%%%%%%%%%%%%%%%%%%%%%%%%%%%%%%%%%%%%%%%%%%%%%%%%%%%%%%%%%%%%%
%%%%%%%%%%%%%%%%%%%%%%%%%%%%%%%%%%%%%%%%%%%%%%%%%%%%%%%%%%%%%%%%%%%%%%%%%%%%%%%%
%%%%%%%%%%%%%%%%%%%%%%%%%%%%%%%%%%%%%%%%%%%%%%%%%%%%%%%%%%%%%%%%%%%%%%%%%%%%%%%%
%%%%%%%%%%%%%%%%%%%%%%%%%%%%%%%%%%%%%%%%%%%%%%%%%%%%%%%%%%%%%%%%%%%%%%%%%%%%%%%%
%%%%%%%%%%%%%%%%%%%%%%%%%%%%%%%%%%%%%%%%%%%%%%%%%%%%%%%%%%%%%%%%%%%%%%%%%%%%%%%%
\startchapter[title={Les entités de base du modèle e-RecolNat+},reference=model:base]

Le modèle e-RecolNat+ repose sur un petit ensemble d'entités abstraites destinées à être déclinées en entités métier et auxquelles elles fournissent des possibilités d'annotation et de structuration socialement et temporellement situées et scientifiquement qualifiées.
Sur la figure \in[f:uml:entities], ces entités sont marquées comme issues du groupe {\tt Base}.
Les concepts présentés ici, qui sont très abstraits, seront exemplifiés en \in[model:tags] et \in[model:discussions].

\fig{Entities}{Diagramme de classes UML simplifié du modèle de base e-RecolNat+}{f:uml:entities}{max}

%%%%%%%%%%%%%%%%%%%%%%%%%%%%%%%%%%%%%%%%%%%%%%%%%%%%%%%%%%%%%%%%%%%%%%%%%%%%%%%%
%%%%%%%%%%%%%%%%%%%%%%%%%%%%%%%%%%%%%%%%%%%%%%%%%%%%%%%%%%%%%%%%%%%%%%%%%%%%%%%%
%%%%%%%%%%%%%%%%%%%%%%%%%%%%%%%%%%%%%%%%%%%%%%%%%%%%%%%%%%%%%%%%%%%%%%%%%%%%%%%%
\startsection[title={Entité abstraite Recolnat (EAR)}]

Le concept d'Entité Recolnat abstraite fournit le type fondamental du modèle e-ReColNat+ dont découle tout objet d'intérêt (à l'exception des ESA, voir \in[model:base:esa]).
Ce type se définit par les caractéristiques intrinsèques et relationnelles suivantes (voir figure \in[f:uml:entities]) :

\startitemize
\item Une EAR possède un identifiant e-ReColNat+ qui lui confère une identité stable au sein du système indépendamment de son type métier concret.
\item La création d'une EAR est située : tout objet sait a quel date, par qui (c'est-à-dire par quel utilisateur e-ReColNat) et dans quel module ou lieu du portail (collaboratoire, visite virtuelle, réseau social, herbonautes, etc.) il a été créé.
\item Une EAR peut être associée à une Entité scientifique abstraite (voir \in[model:base:esa]), et donc se rapporter à quelque chose qui existe en dehors de son module d'origine voire en dehors du système d'information e-ReColNat.
\item Une EAR peut se rapporter à une ou plusieurs EARs cibles (voir lien réflexif {\tt isAbout/relatedEntities} sur le diagramme de la figure \in[f:uml:entities]).
Ce lien dénote le positionnement fondamentalement annotationnel/critique du modèle e-ReColNat+, tout objet pouvant être relié à un autre par une relation hyperdocumentaire située et qualifiée (voir \in[model:base:situation]).
\stopitemize

\stopsection
%%%%%%%%%%%%%%%%%%%%%%%%%%%%%%%%%%%%%%%%%%%%%%%%%%%%%%%%%%%%%%%%%%%%%%%%%%%%%%%%
%%%%%%%%%%%%%%%%%%%%%%%%%%%%%%%%%%%%%%%%%%%%%%%%%%%%%%%%%%%%%%%%%%%%%%%%%%%%%%%%
%%%%%%%%%%%%%%%%%%%%%%%%%%%%%%%%%%%%%%%%%%%%%%%%%%%%%%%%%%%%%%%%%%%%%%%%%%%%%%%%
\startsection[title={Entité scientifique abstraite (ESA)},reference=model:base:esa]

Une Entité scientifique abstraite représente un objet existant dans une base de données extérieure (base agrégée du WP2, ou toute autre base scientifique).
L'identité d'une ESA se compose de deux propriétés : un {\tt type} renvoyant à la base de données scientifique de provenance, et un identifiant {\tt id} à valeur dans cette base\footnote{Il relève de la responsabilité des vues de construire des URLs d'accès effectif aux objets scientifique externes sur la base de ces deux champs.}.
Tout objet extérieur à la base e-ReColNat+ peut ainsi être désigné de manière univoque.

\stopsection
%%%%%%%%%%%%%%%%%%%%%%%%%%%%%%%%%%%%%%%%%%%%%%%%%%%%%%%%%%%%%%%%%%%%%%%%%%%%%%%%
%%%%%%%%%%%%%%%%%%%%%%%%%%%%%%%%%%%%%%%%%%%%%%%%%%%%%%%%%%%%%%%%%%%%%%%%%%%%%%%%
%%%%%%%%%%%%%%%%%%%%%%%%%%%%%%%%%%%%%%%%%%%%%%%%%%%%%%%%%%%%%%%%%%%%%%%%%%%%%%%%
\startsection[title={Situation},reference=model:base:situation]

La classe d'association {\tt Situation} telle qu'elle apparaît sur le diagramme de la figure \in[f:uml:entities] fournit un contexte à la relation {\tt isAbout} réflexive des EARs susmentionnée.
Chaque instance de cette relation est ainsi enrichie en premier lieu d'un type (propriété {\tt relationType}), dénotant la sémantique du lien documentaire ou critique qu'elle établit :

\startitemize
\item lien d'{\em enrichissement} (exemple : une annotation enrichit l'EAR sur laquelle elle porte) ;
\item lien d'{\em approbation} ou à l'inverse de {\em désaccord} (voir {\it infra}) ;
\item lien de {\em qualification} (exemple : un tag qualifie une EAR, voir \in[model:tags]) ;
\item lien d'{\em association} (exemple : un extrait de planche délimité dans le collaboratoire peut être associé à des ressources complémentaires telles que des notes ou des photos, ce qui constitue le fondement d'un carnet botanique hyperdocumentaire et hypermédia) ;
\item etc. (il est possible d'étendre le modèle avec d'autres sémantiques de lien réalisant des opérations critiques non prévues initialement).
\stopitemize
 
Ceci permet de typer les relations entre objets sans sacrifier l'unicité du graphe hyperdocumentaire auquel ils prennent part.
De plus, une Situation est également une EAR (au sens d'une relation UML de généralisation), ce qui confère à sa création la possibilité d'être datée et signée (par un utilisateur identifié).

Par ailleurs, en temps qu'EAR, une Situation jouit d'une identité propre qui lui permet à son tour d'être prise pour objet par la relation {\tt isAbout}.
Cette capacité peut être utilisée pour représenter le souhait d'un utilisateur de qualifier l'action d'un autre utilisateur portant sur la mise en relation de deux EARs.
Ceci appelle un exemple.
Soit un utilisateur U1 --- un aimable contributeur --- qui propose une transcription T1 pour un champ C1 de l'étiquette E1 d'une planche P1.
La réalisation de cette action générera une Situation S1 situant cette action dans le temps (date de création), dans l'espace social (l'auteur U1) et dans l'espace documentaire (mise en relation de T1 à C1 via un lien de type « enrichissement »\footnote{Une sémantique plus précise peut être admise, comme par exemple « Transcription », étant donné qu'il s'agit là d'un concept métier omniprésent.}).
La Situation S1 jouissant d'une identité propre, il devient possible à un autre utilisateur U2 --- un éminent botaniste reconnu comme tel par le composant d'authentification/habilitation --- de donner son avis sur celle-ci, par l'intermédiaire d'un lien de type « Approbation ».
La contribution T1 d'U1 est ainsi marquée comme contribution de premier plan (car reconnue par un chercheur « habilité »), ce qui peut constituer un critère de sélection intéressant lorsqu'il est question de compléter une base de données scientifique à partir des données contributives d'e-ReColNat+.

Plus généralement, le couplage des Situations et des liens typés inter-EARs permet de constituer un réseau critique complexe.

\stopsection
%%%%%%%%%%%%%%%%%%%%%%%%%%%%%%%%%%%%%%%%%%%%%%%%%%%%%%%%%%%%%%%%%%%%%%%%%%%%%%%%
%%%%%%%%%%%%%%%%%%%%%%%%%%%%%%%%%%%%%%%%%%%%%%%%%%%%%%%%%%%%%%%%%%%%%%%%%%%%%%%%
%%%%%%%%%%%%%%%%%%%%%%%%%%%%%%%%%%%%%%%%%%%%%%%%%%%%%%%%%%%%%%%%%%%%%%%%%%%%%%%%
\startsection[title={Sous-types d'EARs}]

Outre les Situations, qui ont un statut particulier du fait de leur fonction d'opérateurs de contextualisation de l'ensemble des opérations de mise en relation permises par le système, les EARs se déclinent en deux sous-types principaux : les Entités feuilles abstraites Recolnat et les Entités composites abstraites Recolnat.
Ces entités, présentées {\it infra}, renvoient au \goto{patron de conception « composite »}[url(http://en.wikipedia.org/wiki/Composite_pattern)].

%%%%%%%%%%%%%%%%%%%%%%%%%%%%%%%%%%%%%%%%%%%%%%%%%%%%%%%%%%%%%%%%%%%%%%%%%%%%%%%%
\startsubsection[title={Entité feuille abstraite Recolnat (EFAR)}]

Une Entité feuille abstraite Recolnat représente une information monadique.
Le diagramme de la figure \in[f:uml:entities] en donne les exemples suivants : commentaire simple, transcription, détermination, nom vernaculaire, coordonnées géographiques.
Dans tous ces cas, le contenu de l'EFAR est indivisible, n'admet aucune sous-partie dotée d'une identité propre et jouissant d'un certain degré d'autonomie.

\stopsubsection
%%%%%%%%%%%%%%%%%%%%%%%%%%%%%%%%%%%%%%%%%%%%%%%%%%%%%%%%%%%%%%%%%%%%%%%%%%%%%%%%
\startsubsection[title={Entité composite abstraite Recolnat (ECAR)}]

Une Entité composite abstraite Recolnat admet un ensemble d'EAR définies comme ses sous-parties.
Les types concrets métier du diagramme de la figure \in[f:uml:entities] héritant d'ECAR illustrent tous cette propriété structurelle : un herbier {\it contient} des planches, une planche {\it se compose de} sous-parties d'intérêt (étiquette, feuille identifiée, etc.), un/e conservateur/trice {\it a la responsabilité} d'un ou plusieurs herbiers, une récolte {\it contient} des spécimens, une collection {\it contient} des herbiers/planches, etc.
Par ailleurs, à l'instar de la relation {\tt isAbout}, le lien d'agrégation {\tt containedEntities} entre une ECAR et un ensemble d'EARs implique une EAR Situation en tant que classe d'association.
Ceci permet la contextualisation et la critique de la création des relations d'appartenance (voir \in[model:base:situation]).

\stopsubsection
\stopsection
\stopchapter
%%%%%%%%%%%%%%%%%%%%%%%%%%%%%%%%%%%%%%%%%%%%%%%%%%%%%%%%%%%%%%%%%%%%%%%%%%%%%%%%
%%%%%%%%%%%%%%%%%%%%%%%%%%%%%%%%%%%%%%%%%%%%%%%%%%%%%%%%%%%%%%%%%%%%%%%%%%%%%%%%
%%%%%%%%%%%%%%%%%%%%%%%%%%%%%%%%%%%%%%%%%%%%%%%%%%%%%%%%%%%%%%%%%%%%%%%%%%%%%%%%
%%%%%%%%%%%%%%%%%%%%%%%%%%%%%%%%%%%%%%%%%%%%%%%%%%%%%%%%%%%%%%%%%%%%%%%%%%%%%%%%
%%%%%%%%%%%%%%%%%%%%%%%%%%%%%%%%%%%%%%%%%%%%%%%%%%%%%%%%%%%%%%%%%%%%%%%%%%%%%%%%
\startchapter[title={Le modèle de tags},reference=model:tags]

\fig{Tags}{Diagramme de classes UML partiel du modèle de tags}{f:uml:entities}{125}

Le caractère générique des entités de base du modèle présentées en \in[model:base] ainsi que l'expressivité des relations qu'elles entretiennent rendent possible la génération aisée de modèles métier « partiels ».
Par la simple dérivation du concept de tag de celui d'EFAR, nous obtenons un puissant modèle de tagging :

\startitemize
\item Tout d'abord, un tag se définit intrinsèquement par un nom, et éventuellement une image.
\item En tant qu'EFAR, un tag peut être mis en relation à une EAR via la relation {\tt isAbout}. Le lien réalisé est alors de type « qualification » (voir \in[model:base:situation]).
\item Chaque association d'un tag à une EAR est contextualisé par une Situation, ce qui permet de connaître pour un couple <Tag, EAR> donné le nombre et l'identité des utilisateurs ayant réalisé/confirmé l'association afférente ainsi que les dates de ces actions. Cette possibilité du modèle est déterminante pour bâtir des interfaces de statistique d'usage des ressources de classement au sein d'une communauté.
\item En tant qu'EAR, un tag peut être mis en relation à n'importe quelle autre EAR sous quelque modalité que ce soit. Pour exemple, il est possible d'associer des tags à un utilisateur sous la modalité relationnelle « favoris », ce qui, une fois les fonctions adéquates implémentées, permettrait à tout utilisateur de sélectionner ses tags les plus utilisés sur sa page personnelle.
\item Remarquons enfin, un peu spéculativement, qu'en tant qu'EAR, un tag peut être taggé, ce qui ouvre le champ à des interfaces d'administration des tags communautaires via des tags « administrateurs ».
\stopitemize

\stopchapter
%%%%%%%%%%%%%%%%%%%%%%%%%%%%%%%%%%%%%%%%%%%%%%%%%%%%%%%%%%%%%%%%%%%%%%%%%%%%%%%%
%%%%%%%%%%%%%%%%%%%%%%%%%%%%%%%%%%%%%%%%%%%%%%%%%%%%%%%%%%%%%%%%%%%%%%%%%%%%%%%%
%%%%%%%%%%%%%%%%%%%%%%%%%%%%%%%%%%%%%%%%%%%%%%%%%%%%%%%%%%%%%%%%%%%%%%%%%%%%%%%%
%%%%%%%%%%%%%%%%%%%%%%%%%%%%%%%%%%%%%%%%%%%%%%%%%%%%%%%%%%%%%%%%%%%%%%%%%%%%%%%%
%%%%%%%%%%%%%%%%%%%%%%%%%%%%%%%%%%%%%%%%%%%%%%%%%%%%%%%%%%%%%%%%%%%%%%%%%%%%%%%%
\startchapter[title={Le cas des discussions}]

%%%%%%%%%%%%%%%%%%%%%%%%%%%%%%%%%%%%%%%%%%%%%%%%%%%%%%%%%%%%%%%%%%%%%%%%%%%%%%%%
%%%%%%%%%%%%%%%%%%%%%%%%%%%%%%%%%%%%%%%%%%%%%%%%%%%%%%%%%%%%%%%%%%%%%%%%%%%%%%%%
%%%%%%%%%%%%%%%%%%%%%%%%%%%%%%%%%%%%%%%%%%%%%%%%%%%%%%%%%%%%%%%%%%%%%%%%%%%%%%%%
\startsection[title={Discussions ciblées & discussions générales}]

Nous proposons d'opérer une distinction entre, d'une part, les {\em discussions générales} abordant des points relatifs à la vie de la communauté, portant sur le fonctionnement d'un module, et d'autre part, les discussions de portée scientifique à propos d'un objet documentaire ou d'un concept scientifique identifié.
À la lumière de cette distinction, nous préconisons que les discussions générales se tiennent dans des composants de type forums classiques (nul besoin d'un développement spécifique pour ces fonctions standards d'animation de communauté).
À l'inverse, les discussions portant sur un objet botanique précis s'apparentent à des formes d'annotations dialogiques dont la valeur scientifique est indiscutable.
Le modèle proposé {\it infra} répond à cette exigence.

\stopsection
%%%%%%%%%%%%%%%%%%%%%%%%%%%%%%%%%%%%%%%%%%%%%%%%%%%%%%%%%%%%%%%%%%%%%%%%%%%%%%%%
%%%%%%%%%%%%%%%%%%%%%%%%%%%%%%%%%%%%%%%%%%%%%%%%%%%%%%%%%%%%%%%%%%%%%%%%%%%%%%%%
%%%%%%%%%%%%%%%%%%%%%%%%%%%%%%%%%%%%%%%%%%%%%%%%%%%%%%%%%%%%%%%%%%%%%%%%%%%%%%%%
\startsection[title={Le modèle de discussions},reference=model:discussions]

\fig{Discussions}{Diagramme de classes UML partiel du modèle de discussions}{f:uml:entities}{125}

Dans les termes du modèle e-ReColNat+, une discussion est un cas particulier d'ECAR, ce qui lui permet d'agréger comme ses sous-partie un ensemble de réponses, qui sont des EFARs.
Étudions les possibilités ouvertes par cette modélisation :

\startitemize
\item En tant qu'EAR, les discussions comme les réponses possèdent un/e utilisateur/trice créateur/trice, une date de création, et gardent trace du module ayant accueilli leur création. Ce dernier point permet non seulement d'amorcer une discussion portant sur un objet botanique dans un module et de la poursuivre dans un autre, mais, sur chaque écran de quelque module que ce soit représentant cette discussion, chaque réponse peut indiquer le module dans lequel elle a été formulée.
\item En tant qu'EAR et par la relation {\tt isAbout}, une discussion peut porter (et dans les faits, porte effectivement) sur une autre EAR (par exemple, une entité botanique).
\item Les discussions étant rattachées à des types concrets métier (planche, spécimen, étiquette, récolteur/trice, mission, etc.), le modèle rend possible des opérations de recherche de discussions par type d'entité, et donc des écrans dans les applications clientes de type « Les dernières discussions portant sur les planches ».
\stopitemize

\stopsection
\stopchapter
\stoppart
\stopcomponent