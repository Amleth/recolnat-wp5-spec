\startchapter[title={Les entités de base du modèle RecolNat+},reference=model:base]

%%%%%%%%%%%%%%%%%%%%%%%%%%%%%%%%%%%%%%%%%%%%%%%%%%%%%%%%%%%%%%%%%%%%%%%%%%%%%%%%
%%%%%%%%%%%%%%%%%%%%%%%%%%%%%%%%%%%%%%%%%%%%%%%%%%%%%%%%%%%%%%%%%%%%%%%%%%%%%%%%
%%%%%%%%%%%%%%%%%%%%%%%%%%%%%%%%%%%%%%%%%%%%%%%%%%%%%%%%%%%%%%%%%%%%%%%%%%%%%%%%
\startsection[title={Introduction}]

Ce chapitre présente les fondements et les entités de base du modèle conceptuel ReColNat augmenté afin de préciser dans quel cadre logique nous pensons les rapports entre les modules dont le développement nous incombe et les autres modules du portail coopératif.
Les informations sur le modèle exposées dans ce document sont donc partielles, et insistent sur l'esprit d'ensemble plus que sur le détail des diverses classes de sous entités métier.

Le modèle RecolNat+ repose sur un petit ensemble d'entités abstraites destinées à être déclinées en entités métier et auxquelles elles fournissent des possibilités d'annotation et de structuration socialement et temporellement situées et scientifiquement qualifiées.
Sur la figure \in[f:uml:entities], ces entités sont marquées comme issues du groupe {\tt Base}.
Les concepts présentés ici, qui sont très abstraits, seront exemplifiés en \in[model:tags] et \in[model:discussions].

\fig{Entities}{Diagramme de classes UML simplifié du modèle de base RecolNat+}{f:uml:entities}{max}

\stopsection
%%%%%%%%%%%%%%%%%%%%%%%%%%%%%%%%%%%%%%%%%%%%%%%%%%%%%%%%%%%%%%%%%%%%%%%%%%%%%%%%
%%%%%%%%%%%%%%%%%%%%%%%%%%%%%%%%%%%%%%%%%%%%%%%%%%%%%%%%%%%%%%%%%%%%%%%%%%%%%%%%
%%%%%%%%%%%%%%%%%%%%%%%%%%%%%%%%%%%%%%%%%%%%%%%%%%%%%%%%%%%%%%%%%%%%%%%%%%%%%%%%
\startsection[title={Entité Recolnat Abstraite (ERA)}]

Le concept d'Entité Recolnat Abstraite fournit le type fondamental du modèle ReColNat+ dont découle tout objet d'intérêt (à l'exception des Entités Scientifiques Abstraites, voir \in[model:base:esa]).
Ce type se définit par les caractéristiques intrinsèques et relationnelles suivantes (voir figure \in[f:uml:entities]) :

\startitemize
\item Une ERA possède un identifiant ReColNat+ qui lui confère une identité stable au sein du système indépendamment de son type métier concret.
\item La création d'une ERA est située : tout objet sait à quel date, par qui (c'est-à-dire par quel utilisateur ReColNat) et dans quel module ou lieu du portail (Collaboratoire, Visites Virtuelles, réseau social, Herbonautes, {\it homepage}, etc.) il a été créé.
\item Une ERA peut être associée à une Entité Scientifique Abstraite (voir \in[model:base:esa]), et donc se rapporter à quelque chose qui existe en dehors de son module d'origine voire en dehors du système d'information ReColNat (n'importe quel document identifiable et accessible sur le Web porteur d'une valeur scientifique).
\item Une ERA peut se rapporter à une ou plusieurs ERAs  cibles (voir lien réflexif {\tt isAbout/relatedEntities} sur le diagramme de la figure \in[f:uml:entities]).
Ce lien dénote le positionnement fondamentalement annotationnel/critique du modèle ReColNat+, tout objet pouvant être relié à un autre par une relation hyperdocumentaire située et qualifiée (voir \in[model:base:opirel]).
\stopitemize

\stopsection
%%%%%%%%%%%%%%%%%%%%%%%%%%%%%%%%%%%%%%%%%%%%%%%%%%%%%%%%%%%%%%%%%%%%%%%%%%%%%%%%
%%%%%%%%%%%%%%%%%%%%%%%%%%%%%%%%%%%%%%%%%%%%%%%%%%%%%%%%%%%%%%%%%%%%%%%%%%%%%%%%
%%%%%%%%%%%%%%%%%%%%%%%%%%%%%%%%%%%%%%%%%%%%%%%%%%%%%%%%%%%%%%%%%%%%%%%%%%%%%%%%
\startsection[title={Entité Scientifique Abstraite (ESA)},reference=model:base:esa]

Une Entité Scientifique Abstraite représente un objet existant dans une base de données extérieure (base agrégée du WP2, ou toute autre base scientifique).
L'identité d'une ESA se compose de deux propriétés : un {\tt type} renvoyant au type de l'objet dans sa base de données scientifique de provenance, et un identifiant {\tt id} à valeur dans cette base\footnote{Il relève de la responsabilité des vues de construire des URLs d'accès effectif aux objets scientifique externes sur la base de ces deux champs.}.
Tout objet extérieur à la base ReColNat+ peut ainsi être désigné de manière univoque.

\stopsection
%%%%%%%%%%%%%%%%%%%%%%%%%%%%%%%%%%%%%%%%%%%%%%%%%%%%%%%%%%%%%%%%%%%%%%%%%%%%%%%%
%%%%%%%%%%%%%%%%%%%%%%%%%%%%%%%%%%%%%%%%%%%%%%%%%%%%%%%%%%%%%%%%%%%%%%%%%%%%%%%%
%%%%%%%%%%%%%%%%%%%%%%%%%%%%%%%%%%%%%%%%%%%%%%%%%%%%%%%%%%%%%%%%%%%%%%%%%%%%%%%%
\startsection[title={Relations & Opinions},reference=model:base:opirel]

Les fonctions annotationnelles et critiques du modèle sont principalement implémentées par deux classes génériques :

\startitemize
\item
La classe {\tt Relationship}, qui, en tant que classe d'association de la relation réflexive {\tt isAbout} des ERAs (cf. diagramme de la figure \in[f:uml:entities]), représente toute relation pouvant exister entre deux entités, quelles qu'elles soient.
Sa propriété {\tt type} précise alors la sémantique de cette relation.
Un tel objet n'exprime pas grand chose en lui-même au-delà du fait qu'il connecte deux ERAs sous une modalité donnée, mais sert de point d'ancrage à une multiplicité d'Opinions venant caractériser ce lien inter-entités (cf. {\it infra}).
La Relation tissée entre deux ERAs peut être, pour exemples :
		\startitemize
		\item un lien de {\em qualification} (exemple : un tag qualifie une ERA, voir \in[model:tags]) ;
		\item un lien d'{\em association} (exemple : un extrait de planche délimité dans le Collaboratoire peut être associé à des ressources complémentaires telles que des notes ou des photos, ce qui constitue le fondement d'un carnet botanique hyperdocumentaire et hypermédia) ;
		\item un lien {\em méréologique} (exemple : une planche fait partie d'une collection).
		\item \dots{}
		\stopitemize
\item
La classe {\tt Opinion}, qui, en tant que classe d'association entre {\tt Relationship} et {\tt User}, représente un engagement critique ou contributif d'un utilisateur à propos d'une ERA.
Une Opinion est {\em signée} (par un utilisateur identifié), {\em datée}, et {\em porteuse d'une intention interprétative}.
Cette intention peut ainsi renvoyer à :
		\startitemize
		\item un lien d'{\em approbation} ou à l'inverse de {\em désaccord} (voir {\it infra}) ;
		\item un lien d'{\em enrichissement} (exemple : une annotation enrichit l'ERA sur laquelle elle porte) ;
		\item \dots{}
		\stopitemize
On remarque qu'à la différence des types de Relations, les types d'Opinions relèvent de l'interprétation, et non du simple établissement d'une relation entre deux objets documentaires.
\stopitemize

Par ailleurs, en temps qu'ERA, une Opinion jouit d'une identité propre qui lui permet à son tour d'être prise pour objet par la relation {\tt isAbout}.
Cette capacité peut être utilisée pour représenter le souhait d'un utilisateur de qualifier l'action d'un autre utilisateur portant sur la mise en relation de deux ERAs.

Ceci appelle un exemple.
Soit un utilisateur U1 --- un aimable contributeur --- qui propose une transcription T1 pour un champ C1 de l'étiquette E1 d'une planche P1.
Cette action générera :
1) une Relation R1 entre T1 et C1 de type « association » (il semble raisonnable de considérer pour l'instant qu'une transcription est « associée » à un champ) ;
2) une Opinion signée et datée de type « approbation » entre U1 et R1.
L'Opinion O1 jouissant d'une identité propre, il devient possible à un autre utilisateur U2 --- un éminent botaniste reconnu comme tel par le service d'authentification/habilitation --- de donner son avis (ici, positif) sur celle-ci, par l'intermédiaire d'un lien de type « Approbation ».
La contribution T1 d'U1 est ainsi marquée comme contribution de premier plan (car reconnue par un chercheur « habilité »), ce qui peut constituer un critère de sélection intéressant lorsqu'il est question de compléter une base de données scientifique à partir des données contributives de ReColNat+.

La dualité Relations/Opinions permet de constituer un réseau critique complexe entre entités documentaires existantes ou contributives sans sacrifier l'unicité du graphe hyperdocumentaire auquel ils prennent part.

\stopsection
%%%%%%%%%%%%%%%%%%%%%%%%%%%%%%%%%%%%%%%%%%%%%%%%%%%%%%%%%%%%%%%%%%%%%%%%%%%%%%%%
%%%%%%%%%%%%%%%%%%%%%%%%%%%%%%%%%%%%%%%%%%%%%%%%%%%%%%%%%%%%%%%%%%%%%%%%%%%%%%%%
%%%%%%%%%%%%%%%%%%%%%%%%%%%%%%%%%%%%%%%%%%%%%%%%%%%%%%%%%%%%%%%%%%%%%%%%%%%%%%%%
\startsection[title={Sous-types d'ERAs}]

Outre les Relations et Opinions, qui ont un statut particulier du fait de leur fonction d'opérateurs de contextualisation de l'ensemble des opérations de mise en relation permises par le système, les ERAs se déclinent en deux sous-types principaux : les entités feuilles abstraites Recolnat et les entités composites abstraites Recolnat.
Ces entités, présentées {\it infra}, renvoient au \goto{patron de conception « composite »}[url(http://en.wikipedia.org/wiki/Composite_pattern)].

%%%%%%%%%%%%%%%%%%%%%%%%%%%%%%%%%%%%%%%%%%%%%%%%%%%%%%%%%%%%%%%%%%%%%%%%%%%%%%%%
\startsubsection[title={Entité feuille abstraite Recolnat (EFAR)}]

Une EFAR représente une information monadique.
Le diagramme de la figure \in[f:uml:entities] en donne les exemples suivants : commentaire simple, transcription, détermination, nom vernaculaire, coordonnées géographiques.
Dans tous ces cas, le contenu de l'EFAR est indivisible, n'admet aucune sous-partie dotée d'une identité propre et jouissant d'un certain degré d'autonomie.

\stopsubsection
%%%%%%%%%%%%%%%%%%%%%%%%%%%%%%%%%%%%%%%%%%%%%%%%%%%%%%%%%%%%%%%%%%%%%%%%%%%%%%%%
\startsubsection[title={Entité composite abstraite Recolnat (ECAR)}]

Une entité composite abstraite Recolnat admet un ensemble d'ERA définies comme ses sous-parties.
Les types concrets métier du diagramme de la figure \in[f:uml:entities] héritant d'ECAR illustrent tous cette propriété structurelle : un herbier {\it contient} des planches, une planche {\it se compose de} sous-parties d'intérêt (étiquette, feuille identifiée, etc.), un/e conservateur/trice {\it a la responsabilité} d'un ou plusieurs herbiers, une récolte {\it contient} des spécimens, une collection {\it contient} des herbiers/planches, etc.
Par ailleurs, à l'instar de la relation {\tt isAbout}, le lien d'agrégation {\tt containedEntities} entre une ECAR et un ensemble d'ERAs implique une ERA Opinion en tant que classe d'association.
Ceci permet la contextualisation et la critique de la création des relations d'appartenance (voir \in[model:base:opirel]).

\stopsubsection
\stopsection
\stopchapter