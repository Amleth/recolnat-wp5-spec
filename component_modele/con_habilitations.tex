\starthiding$

\startchapter[title={Le modèle d'habilitations}]

\startsection[title={Les Emails}]

Un Utilisateur peut appartenir à plusieurs Institutions.
Institutions & Utilisateurs sont des EARs, reliés par un lien {\tt containedEntities} (voir figure \in[f:uml:entities]).
Considérons que e-ReColNat est une Institution, à laquelle sont rattachés par défaut les nouveaux arrivants ne jouissant d'aucun poste chez l'un des partenaires du projet.
La relation d'appartenance d'un Utilisateur à une Institution est caractérisée par une Situation (voir \in[model:base:situation]) de type Affiliation, qui associe à l'individu un Email.
Un Utilisateur peut ainsi avoir plusieurs Emails.
En tant que Situation, l'Email hérite ainsi d'une période de validité.

\stopsection
\startsection[title={Les Habilitations}]

À chaque Email sont associées des Habilitations ({\it Recolnaut}, {\it Administrator}, {\it Scientist}, {\it Curator}).
A minima, un Utilisateur dispose de l'Habilitation {\it Recolnaut}.
Les Habilitations {\it Scientist} et {\it Curator} sont inférées à partir d'une liste d'Emails institutionnels.
L'Habilitation {\it Recolnaut} est valuée ; le nombre associé dénotant la « force contributive » de l'Utilisateur.
La « force contributive » totale d'un Utilisateur est la somme de celles de ses différentes Habilitations de type {\it Recolnaut}.
Si un Email n'est plus utilisé, la « force contributive » associée n'est pas perdue.

\fig{Users}{Diagramme de classes UML partiel du modèle d'habilitations}{f:uml:Users}{150}

\stopsection
\stopchapter

$\stophiding