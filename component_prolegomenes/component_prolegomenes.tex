\environment environment_recolnat
\startcomponent component_prolegomenes

\startpart[title={Prolégomènes}]

\definedescription[terme][hang=2,width=5cm]
 
\startchapter[title={Terminologie}]

\terme{L'écosystème ReColNat+}
L'écosystème « ReColNat augmenté » ou « ReColNat+ »\footnote{Ce nom est initialement apparu lors des réunions avec Julien Husson.} inventorie et organise l'ensemble des concepts, propriétés et relations sur lesquels reposent les modules développés par le laboratoire DICEN/WP5 (collaboratoire, visite virtuelle, portail, et toutes les interfaces de contribution et de navigation documentaires et sociales qui les constituent ou qu'ils supposent).
Le prédicat « augmenté » dénote que notre système d'information a pour enjeu de fournir des strates d'enrichissements supplémentaires aux entités botaniques identifiées dans le modèle de données du WP2\footnote{Qui est notamment nourri par l'IPT du GBIF.}, aux objets des Herbonautes v2, et plus généralement, aux objets métier de l'ensemble des applications liées à ReColNat.
Les « augmentations » afférentes peuvent être d'ordre documentaire et social.
Il contient toutefois des entités qui lui sont propres ; pour exemples, les tags, les discussions, ou encore les entités convoquées dans les visites virtuelles, chacun de ces concepts étant liés à une ou plusieurs entités plus directement botaniques issues des modèles métier des applications du Muséum et des autres partenaires.
Le modèle de données ReColNat+ procède, d'une part, de l'analyse des pratiques et discours de botanistes professionnels et amateurs\footnote{Menée par Lisa Chupin.}, et d'autre part, d'une analyse plus globale des gestes savants portant sur des documents multimédias hyperliés en contexte de travail coopératif mettant en jeu les compétences scientifiques du laboratoire DICEN.

\terme{La base ReColNat+}
La Base de données ReColNat+ réalise la persistance des objets identifiées par le modèle ReColNat+ (les informations constituant leur identité, leurs propriétés intrinsèques, les données qui les définissent et leurs sont directement rattachées, ainsi que les relations qu'elles entretiennent les unes avec les autres).
Différentes APIs (d'extraction et d'écriture) seront proposées pour rendre possible l'exploitation de certaines fonctions de cette base au sein de l'ensemble des modules du portail\footnote{Et peut-être au-delà de ce périmètre, si cela s'avère pertinent.}.

\terme{Recolnaute}
Toute personne ayant créé un compte sur la {\em homepage} ReColNat et intéressée par les fonctions contributives des différents modules.
Le recours à ce terme permet surtout de distinguer les utilisateurs scientifiques affiliés à une institution connue du réseau ReColNat des bénévoles sans ancrage institutionnel.

\startchapter[title={Conventions}]

\fon{} : Indique une entrée du cahier des charges.

\idea{} : Indique une idée ou l'explicitation d'un processus.

\leafa{} : Indique un cas.

\stoppart
\stopcomponent