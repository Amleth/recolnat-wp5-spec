\startchapter[title={Outils de classification}]

\startTODO
\startitemize
	\item Pouvoir construire une taxonomie de manière graphique en connectant des nœuds et en y associant des documents (même principe que l'espace de travail bidimensionnel : les nœuds d'une taxonomie sont autant d'entités-conteneurs).
	\item Pour un spécimen, représenter les évolutions de sa détermination dans le temps et pour chaque déterminateur. Mme X propose une détermination D1 pour le spécimen S1 en 1834 ; M Y propose une détermination D2 pour S1 en 1872, etc.
	\item Un outil de comparaison de graphes pour comparer des classifications concurrentes portant sur le même ensemble de spécimens.
	\item Pouvoir établir que deux termes différents issus de systèmes de classifications différents sont synonymes. Exploiter cette relation de synonymie dans les recherches.
	\item Jeter un oeil sur \goto{APG3 (puis 4, puis 5)}[url(http://en.wikipedia.org/wiki/APG_III_system)].
	\item Jean-Pascal, ex de Tela, pourrait faire du consulting sur les questions de référentiels.
\stopitemize
\stopTODO