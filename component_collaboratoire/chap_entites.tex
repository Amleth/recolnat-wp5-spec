\startchapter[title={Les types d'entités ReColNat+},reference=c:entites]

Dans ce document, le vocable « entité » sera utilisé pour désigner tout objet d'intérêt exposé et offert à la manipulation dans le Collaboratoire.

Ce chapitre dresse la liste des entités identifiées.

%%%%%%%%%%%%%%%%%%%%%%%%%%%%%%%%%%%%%%%%%%%%%%%%%%%%%%%%%%%%%%%%%%%%%%%%%%%%%%%%
%%%%%%%%%%%%%%%%%%%%%%%%%%%%%%%%%%%%%%%%%%%%%%%%%%%%%%%%%%%%%%%%%%%%%%%%%%%%%%%%
%%%%%%%%%%%%%%%%%%%%%%%%%%%%%%%%%%%%%%%%%%%%%%%%%%%%%%%%%%%%%%%%%%%%%%%%%%%%%%%%
%%%%%%%%%%%%%%%%%%%%%%%%%%%%%%%%%%%%%%%%%%%%%%%%%%%%%%%%%%%%%%%%%%%%%%%%%%%%%%%%
%%%%%%%%%%%%%%%%%%%%%%%%%%%%%%%%%%%%%%%%%%%%%%%%%%%%%%%%%%%%%%%%%%%%%%%%%%%%%%%%
\startsection[title={Identification des entités}]

%%%%%%%%%%%%%%%%%%%%%%%%%%%%%%%%%%%%%%%%%%%%%%%%%%%%%%%%%%%%%%%%%%%%%%%%%%%%%%%%
%%%%%%%%%%%%%%%%%%%%%%%%%%%%%%%%%%%%%%%%%%%%%%%%%%%%%%%%%%%%%%%%%%%%%%%%%%%%%%%%
%%%%%%%%%%%%%%%%%%%%%%%%%%%%%%%%%%%%%%%%%%%%%%%%%%%%%%%%%%%%%%%%%%%%%%%%%%%%%%%%
\startsubsection[title={Entités sociales}]

\startitemize
	\item un compte utilisateur inscrit et ses affiliations (institutions, associations savantes, groupes, etc.) & rôles (recolnaute, chercheur, conservateur, gestionnaire de communauté, etc.) associés
	\item une institution
	\item un groupe d'utilisateurs, persistent ou créé de manière {\it ad-hoc}, pour collaborer autour d'un ensemble de contenus
\stopitemize

%%%%%%%%%%%%%%%%%%%%%%%%%%%%%%%%%%%%%%%%%%%%%%%%%%%%%%%%%%%%%%%%%%%%%%%%%%%%%%%%
%%%%%%%%%%%%%%%%%%%%%%%%%%%%%%%%%%%%%%%%%%%%%%%%%%%%%%%%%%%%%%%%%%%%%%%%%%%%%%%%
%%%%%%%%%%%%%%%%%%%%%%%%%%%%%%%%%%%%%%%%%%%%%%%%%%%%%%%%%%%%%%%%%%%%%%%%%%%%%%%%
\startsubsection[title={Entités scientifiques},reference=c:entites:scientifiques]

\startitemize
	\item un récolteur
	\item une récolte
	\item une collection
	\item un herbier
	\item une planche
	\item un spécimen
	\item une détermination à valeur dans un système taxonomique donné (une espèce, un genre, etc.)
	\item une transcription
	\item un nom vernaculaire
\stopitemize

%%%%%%%%%%%%%%%%%%%%%%%%%%%%%%%%%%%%%%%%%%%%%%%%%%%%%%%%%%%%%%%%%%%%%%%%%%%%%%%%
%%%%%%%%%%%%%%%%%%%%%%%%%%%%%%%%%%%%%%%%%%%%%%%%%%%%%%%%%%%%%%%%%%%%%%%%%%%%%%%%
%%%%%%%%%%%%%%%%%%%%%%%%%%%%%%%%%%%%%%%%%%%%%%%%%%%%%%%%%%%%%%%%%%%%%%%%%%%%%%%%
\startsubsection[title={Entités contributives}]

\startitemize
	\item une relation entre deux entités
	\item un opinion au sujet d'une relation
	\item un fil de discussion
	\item une réponse dans un fil de discussion
	\item un détail d'une planche graphiquement localisé (étiquette, etc.) ou d'un spécimen (une feuille, un bourgeon, etc.)
	\item un contenu texte de petite taille (une annotation, une note, etc.)
	\item un Tag
\stopitemize

%%%%%%%%%%%%%%%%%%%%%%%%%%%%%%%%%%%%%%%%%%%%%%%%%%%%%%%%%%%%%%%%%%%%%%%%%%%%%%%%
%%%%%%%%%%%%%%%%%%%%%%%%%%%%%%%%%%%%%%%%%%%%%%%%%%%%%%%%%%%%%%%%%%%%%%%%%%%%%%%%
%%%%%%%%%%%%%%%%%%%%%%%%%%%%%%%%%%%%%%%%%%%%%%%%%%%%%%%%%%%%%%%%%%%%%%%%%%%%%%%%
%%%%%%%%%%%%%%%%%%%%%%%%%%%%%%%%%%%%%%%%%%%%%%%%%%%%%%%%%%%%%%%%%%%%%%%%%%%%%%%%
%%%%%%%%%%%%%%%%%%%%%%%%%%%%%%%%%%%%%%%%%%%%%%%%%%%%%%%%%%%%%%%%%%%%%%%%%%%%%%%%
\startsection[title={Propriétés générales des entités}]

%%%%%%%%%%%%%%%%%%%%%%%%%%%%%%%%%%%%%%%%%%%%%%%%%%%%%%%%%%%%%%%%%%%%%%%%%%%%%%%%
%%%%%%%%%%%%%%%%%%%%%%%%%%%%%%%%%%%%%%%%%%%%%%%%%%%%%%%%%%%%%%%%%%%%%%%%%%%%%%%%
%%%%%%%%%%%%%%%%%%%%%%%%%%%%%%%%%%%%%%%%%%%%%%%%%%%%%%%%%%%%%%%%%%%%%%%%%%%%%%%%
\startsubsection[title={Notion d'accréditation}]

\exig{}
Le système doit pouvoir représenter le fait que certains utilisateurs jouissent d'un contrôle particulier sur certaines entités au regard de certaines opérations du fait que, dans le monde réel, ils tiennent le rôle de gestionnaire des objets que représentent ces entités.

Pour exemple, le responsable d'une entité collection dispose à ce titre d'une accréditation à changer le nom de celle-ci.

\exig{}
Le système doit permettre aux utilisateurs accrédités sur une entité de conférer tout ou partie de leur pouvoir à d'autres utilisateurs.

Pour exemple, le responsable d'une entité collection peut accorder une accréditation à un chercheur pour travailler sur les planches qui sont sous responsabilité.

%%%%%%%%%%%%%%%%%%%%%%%%%%%%%%%%%%%%%%%%%%%%%%%%%%%%%%%%%%%%%%%%%%%%%%%%%%%%%%%%
%%%%%%%%%%%%%%%%%%%%%%%%%%%%%%%%%%%%%%%%%%%%%%%%%%%%%%%%%%%%%%%%%%%%%%%%%%%%%%%%
%%%%%%%%%%%%%%%%%%%%%%%%%%%%%%%%%%%%%%%%%%%%%%%%%%%%%%%%%%%%%%%%%%%%%%%%%%%%%%%%
\startsubsection[title={Métadonnées internes}]

\exig{}
Le système doit permettre de visualiser les métadonnées de chaque entité exposée dans l'interface, tant les métadonnées internes éditables que les métadonnées internes non éditables.
Ces métadonnées sont cruciales pour la compréhension de ce qu'est une entité.

Note :
Le composant d'interface dans lequel sont affichées les métadonnées internes d'une entité est désigné dans ce document comme « l'Inspecteur d'entités » (le nommer est important car il accueillera d'autres fonctions).

%%%%%%%%%%%%%%%%%%%%%%%%%%%%%%%%%%%%%%%%%%%%%%%%%%%%%%%%%%%%%%%%%%%%%%%%%%%%%%%%
\startsubsubsection[title={Métadonnées internes non éditables}]

\exig{}
Le système doit associer à chaque entité des métadonnées internes non éditables caractérisant sa création (son type, le nom du module au sein duquel elle a été créée, la date de création, l'identité de l'entité sociale qui l'a créée).

\exig{}
Le système doit pouvoir référencer n'importe quel objet accessible via le Web (publiquement ou via un accès authentifié ReColNat) susceptible de présenter un intérêt scientifique pour les utilisateurs.
Les entités représentant ces « objets externes » doivent fournir le même niveau d'instrumentation sociale et documentaire que les entités ayant été créées dans ReColNat+.
Ces entités doivent également connaître le type de l'objet externe qu'elles désignent, afin notamment d'informer les interfaces de la conduite à tenir lorsqu'il est question de les représenter et de les manipuler.
Ces entités doivent également recevoir un identifiant à valeur dans la base de données externe dont elles proviennent, afin notamment de pouvoir construire des URLs d'accès.

%%%%%%%%%%%%%%%%%%%%%%%%%%%%%%%%%%%%%%%%%%%%%%%%%%%%%%%%%%%%%%%%%%%%%%%%%%%%%%%%
\startsubsubsection[title={Métadonnées internes éditables}]

\exig{}
Le système doit permettre aux gestionnaires accrédités d'une entité de définir des métadonnées internes éditables (son nom, sa description).

%%%%%%%%%%%%%%%%%%%%%%%%%%%%%%%%%%%%%%%%%%%%%%%%%%%%%%%%%%%%%%%%%%%%%%%%%%%%%%%%
%%%%%%%%%%%%%%%%%%%%%%%%%%%%%%%%%%%%%%%%%%%%%%%%%%%%%%%%%%%%%%%%%%%%%%%%%%%%%%%%
%%%%%%%%%%%%%%%%%%%%%%%%%%%%%%%%%%%%%%%%%%%%%%%%%%%%%%%%%%%%%%%%%%%%%%%%%%%%%%%%
%%%%%%%%%%%%%%%%%%%%%%%%%%%%%%%%%%%%%%%%%%%%%%%%%%%%%%%%%%%%%%%%%%%%%%%%%%%%%%%%
%%%%%%%%%%%%%%%%%%%%%%%%%%%%%%%%%%%%%%%%%%%%%%%%%%%%%%%%%%%%%%%%%%%%%%%%%%%%%%%%
\startsection[title={Éléments de modélisation}]

Le diagramme de la figure \in[f:uml:base] expose les concepts de base du modèle ReColNat+.
Tout objet existant dans ReColNat+ est représenté par la classe {\tt Abstract Entity}.
Ce modèle permet notamment :

\startitemize
	\item
	de représenter des relations structurelles méréologiques (pour exemple, une collection contient des planches, une planche contient des spécimens, etc.) ;
	\item
	de représenter des relations d'association tissées entre entités de diverses natures et dont la sémantique est totalement libre.
\stopitemize

\fig{UML_Base}{Diagramme de classes UML simplifié du modèle de base RecolNat+}{f:uml:base}{max}

Raccourcis utilisés dans ce document (cf. diagramme de la figure \in[f:uml:base]) :

\startitemize
	\item {\tt Abstract Entity} : {\tt AE}
	\item {\tt Abstract Composite Entity} : {\tt ACE}
	\item {\tt Abstract Leaf Entity} : {\tt ALE}
	\item {\tt Abstract Social Entity} : {\tt ASE}
	\item {\tt Scientific Entity} : {\tt ScE}
	\item {\tt Relationship} : {\tt Rel}
	\item {\tt Opinion} : {\tt Opn}
\stopitemize