\startchapter[title={Les types d'entités ReColNat(+)},reference=c:entites]

%%%%%%%%%%%%%%%%%%%%%%%%%%%%%%%%%%%%%%%%%%%%%%%%%%%%%%%%%%%%%%%%%%%%%%%%%%%%%%%%
%%%%%%%%%%%%%%%%%%%%%%%%%%%%%%%%%%%%%%%%%%%%%%%%%%%%%%%%%%%%%%%%%%%%%%%%%%%%%%%%
%%%%%%%%%%%%%%%%%%%%%%%%%%%%%%%%%%%%%%%%%%%%%%%%%%%%%%%%%%%%%%%%%%%%%%%%%%%%%%%%
\startsection[title={Identification des entités d'intérêt}]

%%%%%%%%%%%%%%%%%%%%%%%%%%%%%%%%%%%%%%%%%%%%%%%%%%%%%%%%%%%%%%%%%%%%%%%%%%%%%%%%
\startsubsection[title={Entités sociales}]

\startitemize
	\item un compte utilisateur inscrit et ses affiliations (institutions, associations savantes, groupes, etc.) & rôles (recolnaute, chercheur, conservateur, gestionnaire de communauté, etc.) associés
	\item une institution
	\item un groupe d'utilisateurs, persistent ou créé de manière {\it ad-hoc}, pour collaborer autour d'un ensemble de contenus
\stopitemize

%%%%%%%%%%%%%%%%%%%%%%%%%%%%%%%%%%%%%%%%%%%%%%%%%%%%%%%%%%%%%%%%%%%%%%%%%%%%%%%%
\startsubsection[title={Entités scientifiques},reference=c:entites:scientifiques]

\startitemize
	\item un récolteur
	\item une récolte
	\item une collection
	\item un herbier
	\item une planche
	\item un spécimen
	\item une détermination à valeur dans un système taxonomique donné
	\item une transcription
	\item un nom vernaculaire
\stopitemize

%%%%%%%%%%%%%%%%%%%%%%%%%%%%%%%%%%%%%%%%%%%%%%%%%%%%%%%%%%%%%%%%%%%%%%%%%%%%%%%%
\startsubsection[title={Entités contributives}]

\startitemize
	\item une relation entre deux entités
	\item un opinion au sujet d'une relation
	\item un fil de discussion
	\item une réponse dans un fil de discussion
	\item un détail d'une planche graphiquement localisé (étiquette, etc.) ou d'un spécimen (une feuille, un bourgeon, etc.)
	\item un Tag
\stopitemize

%%%%%%%%%%%%%%%%%%%%%%%%%%%%%%%%%%%%%%%%%%%%%%%%%%%%%%%%%%%%%%%%%%%%%%%%%%%%%%%%
%%%%%%%%%%%%%%%%%%%%%%%%%%%%%%%%%%%%%%%%%%%%%%%%%%%%%%%%%%%%%%%%%%%%%%%%%%%%%%%%
%%%%%%%%%%%%%%%%%%%%%%%%%%%%%%%%%%%%%%%%%%%%%%%%%%%%%%%%%%%%%%%%%%%%%%%%%%%%%%%%
\startsection[title={Propriétés générales des entités}]

La création d'une entité ReColNat+ doit être située ; chaque entité doit ainsi garder trace du module au sein duquel elle a été créée ; de la date à laquelle elle a été créée ; de l'identité de son créateur.

%%%%%%%%%%%%%%%%%%%%%%%%%%%%%%%%%%%%%%%%%%%%%%%%%%%%%%%%%%%%%%%%%%%%%%%%%%%%%%%%
%%%%%%%%%%%%%%%%%%%%%%%%%%%%%%%%%%%%%%%%%%%%%%%%%%%%%%%%%%%%%%%%%%%%%%%%%%%%%%%%
%%%%%%%%%%%%%%%%%%%%%%%%%%%%%%%%%%%%%%%%%%%%%%%%%%%%%%%%%%%%%%%%%%%%%%%%%%%%%%%%
\startsection[title={Relations entre entités},reference=s:entites:relations]

Les fonctions identifiées dans ce document --- tant les fonctions de navigation dans les données existantes que celles visant la création et l'organisation de contenus --- s'appuient sur un ensemble de relations de base entre les différentes entités.
Ces relations doivent pouvoir être créées par l'utilisateur ou définies {\it a priori} lors de l'import de données existantes.
Ainsi :

\startitemize
	\item
	Le système doit pouvoir gérer des relations structurelles de type « contenant/contenu » (pour exemple, une collection contient des planches, une planche contient des spécimens, etc.).
	\item
	Le système doit pouvoir gérer des relations d'association sémantique entre entités sous des modalités associatives libres, sans restriction sur la nature de ces modalités ou sur celle des entités qu'elles connectent.
	\item
	Le système doit pouvoir référencer n'importe quel objet accessible via le Web (publiquement ou via un accès authentifié ReColNat) susceptible de présenter un intérêt scientifique pour les utilisateurs.
	Les entités représentant ces « objets externes » doivent fournir le même niveau d'instrumentation sociale et documentaire que les entités ayant été crées dans ReColNat+.
	Ces entités doivent également connaître le type de l'objet externe qu'elles désignent, afin notamment d'informer les interfaces de la conduite à tenir lorsqu'il est question de les représenter et de les manipuler.
	Il faut également que ces entités soient identifiées par un identifiant à valeur dans la base de données externe dont elles proviennent, afin notamment de pouvoir construire des URLs d'accès.
\stopitemize

%%%%%%%%%%%%%%%%%%%%%%%%%%%%%%%%%%%%%%%%%%%%%%%%%%%%%%%%%%%%%%%%%%%%%%%%%%%%%%%%
%%%%%%%%%%%%%%%%%%%%%%%%%%%%%%%%%%%%%%%%%%%%%%%%%%%%%%%%%%%%%%%%%%%%%%%%%%%%%%%%
%%%%%%%%%%%%%%%%%%%%%%%%%%%%%%%%%%%%%%%%%%%%%%%%%%%%%%%%%%%%%%%%%%%%%%%%%%%%%%%%
\startsection[title={Éléments de modélisation}]

Le diagramme de la figure \in[f:uml:base] expose les concepts de base du modèle ReColNat+.

Tout objet existant dans ReColNat+ est représenté par la classe {\tt Abstract Entity}.

\fig{UML_Base}{Diagramme de classes UML simplifié du modèle de base RecolNat+}{f:uml:base}{max}

Raccourcis utilisés dans ce document :

\startitemize
	\item {\tt Abstract Entity} : {\tt AE}
	\item {\tt Abstract Composite Entity} : {\tt ACE}
	\item {\tt Abstract Leaf Entity} : {\tt ALE}
	\item {\tt Abstract Social Entity} : {\tt ASE}
	\item {\tt Scientific Entity} : {\tt ScE}
	\item {\tt Relationship} : {\tt Rel}
	\item {\tt Opinion} : {\tt Opn}
\stopitemize