\startchapter[title={Espace de travail}]

%%%%%%%%%%%%%%%%%%%%%%%%%%%%%%%%%%%%%%%%%%%%%%%%%%%%%%%%%%%%%%%%%%%%%%%%%%%%%%%%
%%%%%%%%%%%%%%%%%%%%%%%%%%%%%%%%%%%%%%%%%%%%%%%%%%%%%%%%%%%%%%%%%%%%%%%%%%%%%%%%
%%%%%%%%%%%%%%%%%%%%%%%%%%%%%%%%%%%%%%%%%%%%%%%%%%%%%%%%%%%%%%%%%%%%%%%%%%%%%%%%
%%%%%%%%%%%%%%%%%%%%%%%%%%%%%%%%%%%%%%%%%%%%%%%%%%%%%%%%%%%%%%%%%%%%%%%%%%%%%%%%
%%%%%%%%%%%%%%%%%%%%%%%%%%%%%%%%%%%%%%%%%%%%%%%%%%%%%%%%%%%%%%%%%%%%%%%%%%%%%%%%
\startsection[title={Organisation et plasticité de l'espace de travail}]

\idea{}
Nous pensons que l'utilisateur doit pouvoir organiser librement son espace de travail, la synopsis spatiale et la plasticité structurelle de l'environnement documentaire étant déterminantes dans les activités savantes et critiques visant la production de nouvelles connaissances sur la base d'un corpus hétérogène\footnote{Voir par exemple cette \goto{thèse de doctorat}[url(http://amleth.artisiou.com/blog/these)].}.
Cette approche est adaptée aux tâches savantes exploratoires --- comme la classification --- dans lesquelles de nouvelles informations émergent de la manipulation matérielle et critique (comparaison, regroupement, superposition, fragmentation, etc.) des sources documentaires.

\exig{}
Le système doit permettre à l'utilisateur de librement créer des conteneurs pour regrouper des entités, quelles qu'elles soient, et de faire passer à loisir ces entités d'un conteneur à l'autre.
Ces conteneurs sont initialement de simples opérateurs de regroupement, et ne sont pas porteurs d'une sémantique scientifique particulière.

\exig{}
Le système doit permettre à l'utilisateur d'organiser ses conteneurs en arborescence librement modifiables.

\exig{}
Un conteneur doit pouvoir exposer les entités qu'il contient (qu'elles soient des sous-conteneurs ou des entités porteuse d'un contenu textuel ou graphique) au sein d'un espace bidimensionnel, chaque entité y étant représenté sous la forme d'une petite icône librement déplaçable par glisser-déposer dont l'apparence dénote sa nature et/ou son contenu.

\idea{}
Chaque conteneur (ou plutôt, ce qu'il contient) peut ainsi être visualisé sous deux modalités différentes et complémentaires : une modalité arborescente --- pour structurer l'information --- et une modalité spatiale libre --- pour faire émerger des relations entre les entités à mesure que l'activité critique se déploie.

\exig{}
Le système doit permettre aux membres d'un groupe d'utilisateurs souhaitant collaborer autour d'un conteneur de déplacer librement les objets qu'il contient.

%%%%%%%%%%%%%%%%%%%%%%%%%%%%%%%%%%%%%%%%%%%%%%%%%%%%%%%%%%%%%%%%%%%%%%%%%%%%%%%%
%%%%%%%%%%%%%%%%%%%%%%%%%%%%%%%%%%%%%%%%%%%%%%%%%%%%%%%%%%%%%%%%%%%%%%%%%%%%%%%%
%%%%%%%%%%%%%%%%%%%%%%%%%%%%%%%%%%%%%%%%%%%%%%%%%%%%%%%%%%%%%%%%%%%%%%%%%%%%%%%%
%%%%%%%%%%%%%%%%%%%%%%%%%%%%%%%%%%%%%%%%%%%%%%%%%%%%%%%%%%%%%%%%%%%%%%%%%%%%%%%%
%%%%%%%%%%%%%%%%%%%%%%%%%%%%%%%%%%%%%%%%%%%%%%%%%%%%%%%%%%%%%%%%%%%%%%%%%%%%%%%%
\startsection[title={Éléments de modélisation}]

Les conteneurs évoqués {\it supra} sont des {\tt ACE}.

Le fait qu'une {\tt ACE} soit contenue dans une autre se traduit par une {\tt Rel} de type « contenu/contenant ».
Le modèle n'interdit pas qu'une {\tt ACE} possède plusieurs {\tt ACE} parentes, ni même qu'une arborescence d'{\tt ACE} soit circulaire.

Les propriétés de positionnement structurel (numéro d'ordre d'une {\tt AE} au sein de la liste des {\tt AE} enfants de son {\tt AE} parent, position (x, y) d'une {\tt AE} dans la vue espace bidimensionnel de son {\tt AE} parent) sont portées par des {\tt Opn} signées.
Deux scénarios sont alors possibles :

\startitemize
	\item
	Quand un groupe d'utilisateurs souhaite interagir sur la structuration interne d'un conteneur, alors la position d'un de ses enfants est stockée dans une {\tt Opn} unique signée par l'{\tt ASE} qui représente le groupe.
	Cette {\tt Opn} est concurrentielle : chaque utilisateur membre du groupe peut altérer les positions qu'elle porte.
	Dans ce scénario, et si une architecture WebSocket est choisie (voir {\it infra}), l'{\tt ASE} groupe permet au serveur de contrôler le {\it broadcast} des nouveaux déplacements vers les contributeurs intéressés.
	\item
	Quand un groupe d'utilisateurs souhaite laisser ses membres exprimer des avis contraires --- tout en les partageant en lecture seule --- sur la position que doivent occuper des {\tt AE} au sein d'une {\tt ACE}, alors chaque position est portée par un {\tt Opn} signé par un utilisateur identifié.
\stopitemize

%%%%%%%%%%%%%%%%%%%%%%%%%%%%%%%%%%%%%%%%%%%%%%%%%%%%%%%%%%%%%%%%%%%%%%%%%%%%%%%%
%%%%%%%%%%%%%%%%%%%%%%%%%%%%%%%%%%%%%%%%%%%%%%%%%%%%%%%%%%%%%%%%%%%%%%%%%%%%%%%%
%%%%%%%%%%%%%%%%%%%%%%%%%%%%%%%%%%%%%%%%%%%%%%%%%%%%%%%%%%%%%%%%%%%%%%%%%%%%%%%%
%%%%%%%%%%%%%%%%%%%%%%%%%%%%%%%%%%%%%%%%%%%%%%%%%%%%%%%%%%%%%%%%%%%%%%%%%%%%%%%%
%%%%%%%%%%%%%%%%%%%%%%%%%%%%%%%%%%%%%%%%%%%%%%%%%%%%%%%%%%%%%%%%%%%%%%%%%%%%%%%%
\startsection[title={Éléments d'implémentation}]

Pour servir le côté « structure émergente », qui nécessite une interaction à fréquence élevée avec le modèle (un utilisateur peut vouloir modifier plusieurs fois par seconde la position d'un objet) et une synchronisation temps réel des données (un utilisateur collaborant autour d'un espace bidimensionnel doit voir en temps réel les déplacement des sous-entités réalisés par ses collaborateurs), modification des relations structurelles pourrait passer par un serveur WebSocket\footnote{Peut-être \goto{Vert.x}[url(http://vertx.io/)] ?}.