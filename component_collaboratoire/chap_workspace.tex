\startchapter[title={Espace de travail}]

%%%%%%%%%%%%%%%%%%%%%%%%%%%%%%%%%%%%%%%%%%%%%%%%%%%%%%%%%%%%%%%%%%%%%%%%%%%%%%%%
%%%%%%%%%%%%%%%%%%%%%%%%%%%%%%%%%%%%%%%%%%%%%%%%%%%%%%%%%%%%%%%%%%%%%%%%%%%%%%%%
%%%%%%%%%%%%%%%%%%%%%%%%%%%%%%%%%%%%%%%%%%%%%%%%%%%%%%%%%%%%%%%%%%%%%%%%%%%%%%%%
%%%%%%%%%%%%%%%%%%%%%%%%%%%%%%%%%%%%%%%%%%%%%%%%%%%%%%%%%%%%%%%%%%%%%%%%%%%%%%%%
%%%%%%%%%%%%%%%%%%%%%%%%%%%%%%%%%%%%%%%%%%%%%%%%%%%%%%%%%%%%%%%%%%%%%%%%%%%%%%%%
\startsection[title={Organisation et plasticité de l'espace de travail}]

\exig{}
Le système doit permettre à l'utilisateur d'organiser librement son espace de travail\footnote{Le rôle de la plasticité structurelle de l'environnement documentaire dans l'activité de lecture savante visant la production de nouvelles connaissances sur la base d'un corpus hétérogène est exploré dans cette \goto{thèse de doctorat}[url(http://amleth.artisiou.com/blog/these)].} :

\startitemize
	\item
	L'utilisateur doit pouvoir librement créer des conteneurs pour regrouper ses contenus, quels qu'ils soient.
	\item
	L'utilisateur doit pouvoir librement organiser ses conteneurs en arborescence.
	\item
	Un conteneur doit pouvoir exposer son contenu (qu'il s'agisse de sous-conteneurs ou de documents) sous forme de petites icônes librement déplaçables au sein d'un espace bidimensionnel.
	Autrement dit, chaque conteneur (ou plutôt, son contenu) peut être visualisé sous deux modalités différentes et complémentaires : une modalité arborescente --- pour structurer l'information --- et une modalité spatiale libre --- pour faire émerger des relations entre les contenus au fil de l'activité.
	La position de chaque icône au sein de la représentation spatiale de son conteneur doit pouvoir être librement modifiée par glisser-déposer.
	Chaque icône doit avoir une apparence qui dénote sa nature et/ou son contenu.
\stopitemize

\exig{}
Le système doit permettre à un groupe d'utilisateurs souhaitant collaborer autour d'un conteneur de partager les informations de positionnement des objets dans les conteneurs afin de disposer d'un espace d'organisation partagé, les déplacements initiés par chacun étant ainsi vus des autres.

%%%%%%%%%%%%%%%%%%%%%%%%%%%%%%%%%%%%%%%%%%%%%%%%%%%%%%%%%%%%%%%%%%%%%%%%%%%%%%%%
%%%%%%%%%%%%%%%%%%%%%%%%%%%%%%%%%%%%%%%%%%%%%%%%%%%%%%%%%%%%%%%%%%%%%%%%%%%%%%%%
%%%%%%%%%%%%%%%%%%%%%%%%%%%%%%%%%%%%%%%%%%%%%%%%%%%%%%%%%%%%%%%%%%%%%%%%%%%%%%%%
%%%%%%%%%%%%%%%%%%%%%%%%%%%%%%%%%%%%%%%%%%%%%%%%%%%%%%%%%%%%%%%%%%%%%%%%%%%%%%%%
%%%%%%%%%%%%%%%%%%%%%%%%%%%%%%%%%%%%%%%%%%%%%%%%%%%%%%%%%%%%%%%%%%%%%%%%%%%%%%%%
\startsection[title={Éléments de modélisation}]

Les conteneurs évoqués {\it supra} sont des {\tt ACE}.

Le fait qu'une {\tt ACE} soit contenue dans une autre se traduit par une {\tt Rel} de type « contenu/contenant ».
Le modèle n'interdit pas qu'une {\tt ACE} possède plusieurs {\tt ACE} parentes, ni même qu'une arborescence d'{\tt ACE} soit circulaire.

Les propriétés de positionnement structurel (numéro d'ordre d'une {\tt AE} au sein de la liste des {\tt AE} enfants de son {\tt AE} parent, position (x, y) d'une {\tt AE} dans la vue espace bidimensionnel de son {\tt AE} parent) sont portées par des {\tt Opn} signées.
Deux scénarios sont alors possibles :

\startitemize
	\item
	Quand un groupe d'utilisateurs souhaite interagir sur la structuration interne d'un conteneur, alors la position d'un de ses enfants est stockée dans une {\tt Opn} unique signée par l'{\tt ASE} qui représente le groupe.
	Cette {\tt Opn} est concurrentielle : chaque utilisateur membre du groupe peut altérer les positions qu'elle porte.
	Dans ce scénario, et si une architecture WebSocket est choisie (voir {\it infra}), l'{\tt ASE} groupe permet au serveur de contrôler le {\it broadcast} des nouveaux déplacements vers les contributeurs intéressés.
	\item
	Quand un groupe d'utilisateurs souhaite laisser ses membres exprimer des avis contraires --- tout en les partageant en lecture seule --- sur la position que doivent occuper des {\tt AE} au sein d'une {\tt ACE}, alors chaque position est portée par un {\tt Opn} signé par un utilisateur identifié.
\stopitemize

%%%%%%%%%%%%%%%%%%%%%%%%%%%%%%%%%%%%%%%%%%%%%%%%%%%%%%%%%%%%%%%%%%%%%%%%%%%%%%%%
%%%%%%%%%%%%%%%%%%%%%%%%%%%%%%%%%%%%%%%%%%%%%%%%%%%%%%%%%%%%%%%%%%%%%%%%%%%%%%%%
%%%%%%%%%%%%%%%%%%%%%%%%%%%%%%%%%%%%%%%%%%%%%%%%%%%%%%%%%%%%%%%%%%%%%%%%%%%%%%%%
%%%%%%%%%%%%%%%%%%%%%%%%%%%%%%%%%%%%%%%%%%%%%%%%%%%%%%%%%%%%%%%%%%%%%%%%%%%%%%%%
%%%%%%%%%%%%%%%%%%%%%%%%%%%%%%%%%%%%%%%%%%%%%%%%%%%%%%%%%%%%%%%%%%%%%%%%%%%%%%%%
\startsection[title={Éléments d'implémentation}]

Pour servir le côté « structure émergente », qui nécessite une interaction à fréquence élevée avec le modèle (un utilisateur peut vouloir modifier plusieurs fois par seconde la position d'un objet) et une synchronisation temps réel des données (un utilisateur collaborant autour d'un espace bidimensionnel doit voir en temps réel les déplacement des sous-entités réalisés par ses collaborateurs), modification des relations structurelles pourrait passer par un serveur WebSocket\footnote{Peut-être \goto{Vert.x}[url(http://vertx.io/)] ?}.