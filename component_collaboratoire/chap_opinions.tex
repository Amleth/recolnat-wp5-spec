\startchapter[title={Opinions & autorité}]

%%%%%%%%%%%%%%%%%%%%%%%%%%%%%%%%%%%%%%%%%%%%%%%%%%%%%%%%%%%%%%%%%%%%%%%%%%%%%%%%
%%%%%%%%%%%%%%%%%%%%%%%%%%%%%%%%%%%%%%%%%%%%%%%%%%%%%%%%%%%%%%%%%%%%%%%%%%%%%%%%
%%%%%%%%%%%%%%%%%%%%%%%%%%%%%%%%%%%%%%%%%%%%%%%%%%%%%%%%%%%%%%%%%%%%%%%%%%%%%%%%
%%%%%%%%%%%%%%%%%%%%%%%%%%%%%%%%%%%%%%%%%%%%%%%%%%%%%%%%%%%%%%%%%%%%%%%%%%%%%%%%
%%%%%%%%%%%%%%%%%%%%%%%%%%%%%%%%%%%%%%%%%%%%%%%%%%%%%%%%%%%%%%%%%%%%%%%%%%%%%%%%
\startsection[title={Enjeux}]

Pour ce qui se rapporte à la collecte d'informations, l'écosystème ReColNat+ poursuit un objectif double :

\startitemize
	\item
	Permettre aux chercheurs naturalistes de saisir des informations relatives aux entités scientifiques sur lesquelles se porte leur intérêt (cf. \in[c:entites:scientifiques]) dans le cadre de leurs travaux.
	Il s'agit ici de permettre une saisie simple et une dissémination rapide au sein de la communauté scientifique ReColNat des résultats de recherche.
	\item
	Permettre aux recolnautes bénévoles de contribuer aux différentes occasions de récolte de données participatives ouvertes dans le système (transcriptions, noms vernaculaires, diverses informations non scientifiques sur les espèces, identification et localisation de caractères sur les planches, etc.).
	Cette collecte ouverte doit être suivie d'un processus de sélection de l'opinion la plus pertinente.
\stopitemize

Dans ces deux cas, l'information collectée est plurielle, au sens où une multiplicité d'opinions potentiellement contradictoires peuvent être formulées au sujet d'une même chose par des individus différents, ou au regard du fait qu'un même individu est libre de réviser son opinion à la lumière d'informations nouvelles.
Ce chapitre présente les bases d'un système de contribution multi opinions et de sélection de l'opinion pertinente.

%%%%%%%%%%%%%%%%%%%%%%%%%%%%%%%%%%%%%%%%%%%%%%%%%%%%%%%%%%%%%%%%%%%%%%%%%%%%%%%%
%%%%%%%%%%%%%%%%%%%%%%%%%%%%%%%%%%%%%%%%%%%%%%%%%%%%%%%%%%%%%%%%%%%%%%%%%%%%%%%%
%%%%%%%%%%%%%%%%%%%%%%%%%%%%%%%%%%%%%%%%%%%%%%%%%%%%%%%%%%%%%%%%%%%%%%%%%%%%%%%%
%%%%%%%%%%%%%%%%%%%%%%%%%%%%%%%%%%%%%%%%%%%%%%%%%%%%%%%%%%%%%%%%%%%%%%%%%%%%%%%%
%%%%%%%%%%%%%%%%%%%%%%%%%%%%%%%%%%%%%%%%%%%%%%%%%%%%%%%%%%%%%%%%%%%%%%%%%%%%%%%%
\startsection[title={Exigences}]

Cette section emploie les vocables de {\em contribution} et {\em opinions}, qui seront plus formellement définis et illustrés en \in[c:relopn:s:model] ; pour l'instant, entendons par contribution une donnée produite par un utilisateur identifié à un moment donné dans un module donné, que cette donnée se rapporte à une autre donnée (par exemple : une proposition de transcription d'un champ) ou non (par exemple : une note créée {\it ex-nihilo}).

%%%%%%%%%%%%%%%%%%%%%%%%%%%%%%%%%%%%%%%%%%%%%%%%%%%%%%%%%%%%%%%%%%%%%%%%%%%%%%%%
%%%%%%%%%%%%%%%%%%%%%%%%%%%%%%%%%%%%%%%%%%%%%%%%%%%%%%%%%%%%%%%%%%%%%%%%%%%%%%%%
%%%%%%%%%%%%%%%%%%%%%%%%%%%%%%%%%%%%%%%%%%%%%%%%%%%%%%%%%%%%%%%%%%%%%%%%%%%%%%%%
\startsubsection[title={Représentation des contributeurs}]

%%%%%%%%%%%%%%%%%%%%%%%%%%%%%%%%%%%%%%%%%%%%%%%%%%%%%%%%%%%%%%%%%%%%%%%%%%%%%%%%
\startsubsubsection[title={Compte scientifique & rang contributif}]

\exig{}
Le système doit identifier les comptes utilisateurs créés pour des chercheurs.
Cette information doit être exposée dans l'IHM à côté du nom de l'utilisateur.

\idea{}
Le WP5 a besoin d'une liste blanche des chercheurs connus au sein de ReColNat.

\exig{}
À chaque recolnaute doit être associé un rang de contribution, qui augmente de 1 unité à chaque nouvelle contribution.
Cette information doit être exposée dans l'IHM à côté du nom de l'utilisateur.

\idea{}
Faut-il définir un rang de contribution pour les chercheurs ?

\exig{}
À chaque recolnaute doit être associé un rang d'autorité, égal à la somme des contributions effectuées approuvées par un chercheur ou par un utilisateur accrédité\footnote{Sous-entendu : accrédité pour l'entité sur laquelle porte la contribution au moment où cette contribution a été faite.}.
Cette information doit être exposée dans l'IHM à côté du nom de l'utilisateur.

\idea{}
Nul besoin de définir un rang d'autorité pour les chercheurs : elle est reconnue {\it a priori}.

\idea{}
Autres points à discuter :

\startitemize
	\item Un recolnaute doit-il recevoir la même quantité de points d'autorité s'il propose une bonne transcription, une bonne localisation géographique, un bon nom vernaculaire, voire quelque chose de scientifiquement plus consistant (une bonne détermination) ?
	\item La création d'une contribution (qui sera jugée pertinente lors du processus de sélection) ne doit-elle pas apporter plus de points d'autorité que l'approbation d'une contribution (qui elle aussi sera jugée pertinente lors du processus de sélection) ?
\stopitemize

\exig{}
Le système ne doit pas pénaliser les recolnautes ayant émis des opinions disqualifiées par les chercheurs :)

%%%%%%%%%%%%%%%%%%%%%%%%%%%%%%%%%%%%%%%%%%%%%%%%%%%%%%%%%%%%%%%%%%%%%%%%%%%%%%%%
\startsubsubsection[title={Notion d'affiliation}]

\exig{}
Le système doit permettre d'associer des affiliations aux utilisateurs.
Les affiliations sont exposées dans l'IHM sous forme de sceaux graphiques à côté du nom de l'utilisateur.
Lorsqu'un utilisateur crée une nouvelle contribution, il doit pouvoir décider de mettre en jeu ou non chacune de ses contributions.

Pour exemple, un chercheur membre du MNHN et membre de l'association des amateurs de {\it Sphagna squarrosa} du Morvan (la fameuse AASSM) doit pouvoir, lorsqu'il produit une nouvelle Opinion, choisir d'associer à celle-ci :

\startitemize
	\item Le Sceau du MNHN, donnant ainsi une portée institutionnelle à sa contribution.
	\item Le Sceau de l'AASSM.
	\item Les deux Sceaux sus-mentionnés.
	\item Aucun Sceau, faisant ainsi de sa contribution quelque chose de purement individuel et n'engageant aucun collectif.
\stopitemize

%%%%%%%%%%%%%%%%%%%%%%%%%%%%%%%%%%%%%%%%%%%%%%%%%%%%%%%%%%%%%%%%%%%%%%%%%%%%%%%%
%%%%%%%%%%%%%%%%%%%%%%%%%%%%%%%%%%%%%%%%%%%%%%%%%%%%%%%%%%%%%%%%%%%%%%%%%%%%%%%%
%%%%%%%%%%%%%%%%%%%%%%%%%%%%%%%%%%%%%%%%%%%%%%%%%%%%%%%%%%%%%%%%%%%%%%%%%%%%%%%%
\startsubsection[title={Représentation des contributions},reference:s:contributions]

%%%%%%%%%%%%%%%%%%%%%%%%%%%%%%%%%%%%%%%%%%%%%%%%%%%%%%%%%%%%%%%%%%%%%%%%%%%%%%%%
\startsubsubsection[title={Opinions multiples}]

\exig{}
Le système doit permettre à n'importe quel utilisateur de proposer une contribution pour n'importe quelle donnée de la base ReColNat+ exposée dans le Collaboratoire.

\exig{}
Le système doit permettre à n'importe quel utilisateur d'exprimer son opinion au sujet de n'importe quelle contribution réalisée par un autre utilisateur.
Ainsi, chaque utilisateur peur approuver, désapprouver ou exprimer son doute au sujet de toute contribution.

\exig{}
Le système doit permettre à un utilisateur de modifier ou de supprimer une de ses contributions à la condition que celle-ci n'ait pas été prise pour cible par une opinion émise par un autre utilisateur.
Il s'agit ici, par exemple, d'éviter qu'une opinion soit modifiée après qu'elle ait été critiquée (une bonne manière de faire est de créer une nouvelle opinion).

\exig{}
Le système doit maintenir une séparation nette dans son interface entre les contributions/opinions créées par des chercheurs et celles créées par des recolnautes.
Cependant, lorsqu'une contribution/opinion créée par un recolnaute est approuvée par un chercheur, elle est mêlée au groupe des contribution/opinion créées par des chercheurs.
Au sein de ces deux groupes, les contributions/opinions doivent être triées par pertinence (voir {\em infra}).

\exig{}
Le système doit permettre aux utilisateurs d'attribuer un statut public/privé à leurs contributions et opinions, afin, respectivement, de les rendre visibles à tous ou de les garder pour eux.
Une contribution ou opinion privée n'est jamais exposée à la vue des autres utilisateurs que son auteur, n'entre jamais en compte dans un calcul d'autorité et n'est jamais exposée via une API.

%%%%%%%%%%%%%%%%%%%%%%%%%%%%%%%%%%%%%%%%%%%%%%%%%%%%%%%%%%%%%%%%%%%%%%%%%%%%%%%%
%%%%%%%%%%%%%%%%%%%%%%%%%%%%%%%%%%%%%%%%%%%%%%%%%%%%%%%%%%%%%%%%%%%%%%%%%%%%%%%%
%%%%%%%%%%%%%%%%%%%%%%%%%%%%%%%%%%%%%%%%%%%%%%%%%%%%%%%%%%%%%%%%%%%%%%%%%%%%%%%%
\startsubsection[title={Pertinence des contributions}]

\exig{}
Le système doit attribuer à chaque contribution un premier rang de pertinence à valeur dans la communauté des chercheurs.
Ce rang augmente d'une unité lorsque sa contribution est approuvée par un chercheur.
Ce rang n'est pas modifié par les opinions des recolnautes : il instrumente le consensus et le dissensus au sein de la communauté scientifique.

\exig{}
Le système doit attribuer à chaque contribution un second rang de pertinence à valeur dans la communauté des recolnautes.
Lorsqu'un recolnaute approuve (désapprouve) une contribution, le rang de pertinence de la contribution augmente (diminue) d'un nombre égal au rang d'autorité du recolnaute.

\idea{} Ce dernier point est à expérimenter et à discuter.

\exig{}
Le système doit afficher les opinions existantes en distinguant celles émises par des comptes utilisateurs de chercheurs de celles émises par des comptes de recolnautes.
Au sein de chacune de ces catégories, les contributions doivent être classées selon le rang de pertinence.

%%%%%%%%%%%%%%%%%%%%%%%%%%%%%%%%%%%%%%%%%%%%%%%%%%%%%%%%%%%%%%%%%%%%%%%%%%%%%%%%
\startsubsubsection[title={En dehors de R+}]

\exig{}
Le système doit à tout moment être en mesure de fournir au système d'information du WP2 la donnée la plus pertinente au sujet d'une information scientifique ouverte à la contribution, ainsi que des informations sur son contexte de création (identité de l'auteur, date et module de création, etc.).
La donnée ayant le plus fort rang de pertinence parmi les contributions signées ou approuvées par des chercheurs est sélectionnée.
Si plusieurs contributions ont un même rang de pertinence existent, celle ayant été la plus récemment créée ou approuvée est sélectionnée\footnote{Ce mécanisme peut donner lieu à des situations d'{\it edition war}.}.

\idea{}
Si aucune contribution signée ou approuvée par un chercheur n'est disponible, faut-il fournir au SI du WP2 une contribution signée d'un recolnaute ?

%%%%%%%%%%%%%%%%%%%%%%%%%%%%%%%%%%%%%%%%%%%%%%%%%%%%%%%%%%%%%%%%%%%%%%%%%%%%%%%%
%%%%%%%%%%%%%%%%%%%%%%%%%%%%%%%%%%%%%%%%%%%%%%%%%%%%%%%%%%%%%%%%%%%%%%%%%%%%%%%%
%%%%%%%%%%%%%%%%%%%%%%%%%%%%%%%%%%%%%%%%%%%%%%%%%%%%%%%%%%%%%%%%%%%%%%%%%%%%%%%%
%%%%%%%%%%%%%%%%%%%%%%%%%%%%%%%%%%%%%%%%%%%%%%%%%%%%%%%%%%%%%%%%%%%%%%%%%%%%%%%%
%%%%%%%%%%%%%%%%%%%%%%%%%%%%%%%%%%%%%%%%%%%%%%%%%%%%%%%%%%%%%%%%%%%%%%%%%%%%%%%%
\startsection[title={Éléments de modélisation},reference=c:relopn:s:model]

%%%%%%%%%%%%%%%%%%%%%%%%%%%%%%%%%%%%%%%%%%%%%%%%%%%%%%%%%%%%%%%%%%%%%%%%%%%%%%%%
%%%%%%%%%%%%%%%%%%%%%%%%%%%%%%%%%%%%%%%%%%%%%%%%%%%%%%%%%%%%%%%%%%%%%%%%%%%%%%%%
%%%%%%%%%%%%%%%%%%%%%%%%%%%%%%%%%%%%%%%%%%%%%%%%%%%%%%%%%%%%%%%%%%%%%%%%%%%%%%%%
\startsubsection[title={Contribution = Relation + Opinion}]

Ainsi qu'en témoigne le diagramme de la figure \in[f:uml:base], une contribution articule un réseau d'entités :

\startitemize
	\item
	Une {\tt AE} au sujet de laquelle la contribution est faire (par exemple, un champ à transcrire).
	\item
	Une {\tt Rel} porteuse de la donnée contributive (par exemple, une proposition de transcription) et d'une sémantique qui qualifie son rapport à l'{\tt AE} à laquelle elle se rapporte (par exemple : « ceci est une transcription »).
	Remarquons qu'une {\tt Rel} ne dénote en elle-même aucun contexte social ou temporel : elle ne fait que connecter une donnée contributive à une entité sous une certaine sémantique.
	\item
	Une {\tt Opn}, qui date, qualifie (par exemple : approbation, désapprobation, doute) et signe la {\tt Rel}.
	Une {\tt Opn} peut être créée par n'importe quel type d'{\tt ASE} : un utilisateur, un groupe, une institution.
	Les éventuels sceaux sont associés à l'{\tt Opn} en plus de l'identité de son {\tt ASE} créatrice.
\stopitemize

Voici un exemple de scénario de contribution autour de la transcription d'un champ.
Soient un utilisateur U1 proposant une transcription T1 pour le champ C1 de l'étiquette E1 de la planche P1 (note : C1, E1 et P1 sont des {\tt AE} entretenant des {\tt Rel} d'inclusion : P1 contient E1 qui contient C1), un utilisateur U2 proposant une transcription T2 pour C1 (T2 étant différente de T1) & un utilisateur U3, arrivant après U1, apportant son approbation à T1.
Dans ce scénario, deux Relations (R1 connectant T1 à C1 et R2 connectant T2 à C1) et trois Opinions (U1 et U3 à propos de R1, U2 à propos de R2) sont constituées.

\bigskip

Et maintenant, un peu de {\it brainfuck}.

Si un utilisateur émet une opinion de désapprobation au sujet d'une opinion d'approbation portant sur une relation, est-ce la même chose que s'il avait émis une opinion de désapprobation sur cette relation ?

{\it A priori}, non : dans un cas, les arguments qui approuvent la relation sont désapprouvés, dans l'autre, la relation est désapprouvée.