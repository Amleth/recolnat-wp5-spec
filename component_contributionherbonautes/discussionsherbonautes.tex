\startchapter[title={Les discussions},reference=disc]

%%%%%%%%%%%%%%%%%%%%%%%%%%%%%%%%%%%%%%%%%%%%%%%%%%%%%%%%%%%%%%%%%%%%%%%%%%%%%%%%
%%%%%%%%%%%%%%%%%%%%%%%%%%%%%%%%%%%%%%%%%%%%%%%%%%%%%%%%%%%%%%%%%%%%%%%%%%%%%%%%
%%%%%%%%%%%%%%%%%%%%%%%%%%%%%%%%%%%%%%%%%%%%%%%%%%%%%%%%%%%%%%%%%%%%%%%%%%%%%%%%
\startsection[title={Ancrage des discussions},reference=disc:ancrage]

Les Discussions --- au sens du modèle conceptuel qui a été présenté {\em supra} --- doivent pouvoir être créées à trois niveaux :

\startitemize
\item {\em\bf Au niveau des Missions.}
Les Discussions rattachées à une Mission portent sur l'esprit de celle-ci, sur les spécificités de ses consignes, ou sur tout autre point portant sur plusieurs des Spécimens qui la composent.
\item {\em\bf Au niveau des Spécimens.}
Les Discussions rattachées à un Spécimen portent typiquement sur l'objet botanique présent sur la planche ou sur son étiquette en général, sans viser un Champ à transcrire en particulier.
\item {\em\bf Au niveau des Champs à transcrire.}
Chaque Champ à transcrire peut faire l'objet de Discussions lui étant propres, c'est-à-dire ne faisant pas référence au Spécimen sur lequel il est présent ou à la Mission dans le cadre de laquelle ce Spécimen s'inscrit.
Remarquons que les Discussions liées à un Champs à transcrire sont bien rattachées à celui-ci et non sur aux différentes transcriptions proposées par les Herbonautes.
\stopitemize

%%%%%%%%%%%%%%%%%%%%%%%%%%%%%%%%%%%%%%%%%%%%%%%%%%%%%%%%%%%%%%%%%%%%%%%%%%%%%%%%
%%%%%%%%%%%%%%%%%%%%%%%%%%%%%%%%%%%%%%%%%%%%%%%%%%%%%%%%%%%%%%%%%%%%%%%%%%%%%%%%
%%%%%%%%%%%%%%%%%%%%%%%%%%%%%%%%%%%%%%%%%%%%%%%%%%%%%%%%%%%%%%%%%%%%%%%%%%%%%%%%
\startsection[title={Caractérisation des messages & adresse à la communauté}]

%%%%%%%%%%%%%%%%%%%%%%%%%%%%%%%%%%%%%%%%%%%%%%%%%%%%%%%%%%%%%%%%%%%%%%%%%%%%%%%%
\startsubsection[title={Catégorisation des Messages},reference=disc:categorie]

Les Discussions constituent le lieu privilégié où adresser une demande d'aide ou de renseignements à la communauté.
Par ailleurs, elles sont aussi le lieu où des connaissances nouvelles sont apportées par les herbonautes.
Comme vu {\em supra}, chaque Messages/Réponses peut être associé à une ou plusieurs Catégories dénotant la nature de la demande ou contribution qu'il véhicule.
Nous proposons les Catégories de suivantes :

\startitemize
\item message de demande d'aide portant sur les consignes
\item message de demande d'aide au déchiffrage
\item message de demande d'aide à l'identification
\item message de demande de validation/confirmation
\item message apportant une preuve
\item message pointant vers une source documentaire
\stopitemize

Un Message peut être créé sans Catégorie (c'est d'ailleurs le cas le plus fréquent lors d'un échange entre herbonautes). Tout herbonaute peut qualifier un Message existant en lui apposant des Catégories.

Les Catégories permettent de construire des vues générales de tous les messages filtrés par type (\in[disc:recherche]).

%%%%%%%%%%%%%%%%%%%%%%%%%%%%%%%%%%%%%%%%%%%%%%%%%%%%%%%%%%%%%%%%%%%%%%%%%%%%%%%%
\startsubsection[title={Accroître la visibilité d'une Discussion},reference:disc:vis]

Par défaut, les Discussions vivent dans l'espace de la page de l'objet métier auquel elles sont rattachées (Mission, Spécimen, Champ à transcrire).
Toutefois, un-e herbonaute peut souhaiter conférer une visibilité plus importante à une Discussion, notamment s'il s'agit d'une demande d'aide afin de maximiser ses chances d'avoir une réponse.

Chaque Discussion doit ainsi :

\startitemize
\item Pouvoir être marquée comme devant également apparaître sur les pages des objets métier de « plus haut niveau » que celui auquel elle est rattaché. Pour exemple : une discussion rattachée à un Spécimen peut être marquée pour apparaître sur la page de la Mission, une discussion rattachée à un Champ à transcrire peut être marquée pour apparaître sur la page de son Spécimen et de la Mission de celui-ci.
\item Pouvoir être marquée comme devant apparaître dans l'espace des questions non résolues (\in[disc:espaceqnr]).
\stopitemize

Sur les pages de Mission ou de Spécimen, les Discussions directement rattachées à l'objet consulté et celles qui ont été remontées des nivaux inférieurs afin d'accroître leur visibilité devront être visuellement séparées.

\starthiding$
\startnarrower[3cm]
{\em
Quand un cœur volontaire s'abîme dans le doute,\crlf
Quand les lettres et feuilles se font énigmatiques,\crlf
Jaillit alors l'icône enrayant la déroute,\crlf
Et le monde peut ainsi entendre sa supplique.
}
\stopnarrower
$\stophiding

%%%%%%%%%%%%%%%%%%%%%%%%%%%%%%%%%%%%%%%%%%%%%%%%%%%%%%%%%%%%%%%%%%%%%%%%%%%%%%%%
\startsubsection[title={Adresse d'un Message à un-e herbonaute}]

En plus de pouvoir bénéficier d'une visibilité accrue, chaque Message doit pouvoir être adressé :

\startitemize
\item au chef de Mission, ce qui a pour effet de notifier celui-ci via la messagerie interne de l'application (voir {\em infra}) ;
\item à un-e herbonaute en particulier dont l'identifiant interne au site Herbonautes est connu (même processus que ci-dessus avec le chef de mission).
\stopitemize

%%%%%%%%%%%%%%%%%%%%%%%%%%%%%%%%%%%%%%%%%%%%%%%%%%%%%%%%%%%%%%%%%%%%%%%%%%%%%%%%
\startsubsection[title={Demandes d'aide stéréotypés}]

Afin de faire gagner du temps aux herbonautes se livrant à l'activité de transcription et de faciliter la circulation des connaissances, le système doit rendre possible la création de « demandes d'aide stéréotypées ».
Certaines Catégories identifiées en \in[disc:categorie]) sont éligibles à ce principe :

\startitemize
\item demande de validation
\item demande d'aide au déchiffrage
\item demande d'aide à l'identification
\stopitemize

Dans l'interface, chaque Champs à transcrire est équipé d'autant de boutons permettant, en un clic, de créer une nouvelle Disc+ussion incarnant la demande d'aide et contenant un premier Message, lequel est automatiquement titré selon la nature de la demande (c'est-à-dire selon la Catégorie) et l'identité de l'objet métier sur lequel elle porte.

Par ailleurs, les Discussions créées suite à un clic sur ces boutons sont automatiquement rendues visibles aux niveaux de la Mission et de l'espace des Discussions non résolues.

%%%%%%%%%%%%%%%%%%%%%%%%%%%%%%%%%%%%%%%%%%%%%%%%%%%%%%%%%%%%%%%%%%%%%%%%%%%%%%%%
%%%%%%%%%%%%%%%%%%%%%%%%%%%%%%%%%%%%%%%%%%%%%%%%%%%%%%%%%%%%%%%%%%%%%%%%%%%%%%%%
%%%%%%%%%%%%%%%%%%%%%%%%%%%%%%%%%%%%%%%%%%%%%%%%%%%%%%%%%%%%%%%%%%%%%%%%%%%%%%%%
\startsection[title={L'espace des questions non résolues},reference=disc:espaceqnr]

Le système devra proposer un espace des questions non résolues où sera canalisée l'énergie contributive des herbonautes les plus savants, agrégeant l'ensemble des demandes adressées à la communauté ayant émergé lors de l'activité de transcription.

%%%%%%%%%%%%%%%%%%%%%%%%%%%%%%%%%%%%%%%%%%%%%%%%%%%%%%%%%%%%%%%%%%%%%%%%%%%%%%%%
\startsubsection[title={Contextualisation des Discussions}]

Une Discussion exposée dans l'espace des questions non résolue doit être flanquée des informations issues du contexte dont elle provient (Mission, Spécimen, Champ à transcrire, valeurs transcrites) permettant de guider les contributeurs.
Ainsi, pour chaque Discussion, les informations suivantes doivent apparaître :

\startitemize
\item Mission
\item Spécimen (si la Discussion n'est pas rattachée à une Mission)
\item Champ (si la Discussion n'est pas rattachée à une Mission ou un Spécimen)
\item auteur-e à l'origine de la demande
\item titre
\item si la Discussion porte sur un Champ :
    \startitemize
    \item ensemble des valeurs transcrites déjà soumises par les autres herbonautes pour ce Champ
    \item valeurs déjà transcrites des autres Champs, lesquelles fournissent des opérateurs de contextualisation idéals :
        \startitemize
        \item pays/région + géolocalisation exacte déduite de la valeur du champ lieu de récolte transcrit
        \item année/période/date
        \item récolteur
        \item « déterminateur »
        \item collection
        \stopitemize
    \stopitemize
\item Catégorie (\in[disc:categorie])
\item marqueur d'urgence calculé automatiquement en fonction de la date courante et de la date de fin de la Mission
\stopitemize

%%%%%%%%%%%%%%%%%%%%%%%%%%%%%%%%%%%%%%%%%%%%%%%%%%%%%%%%%%%%%%%%%%%%%%%%%%%%%%%%
\startsubsection[title={Résolution d'une Discussion},reference=disc:resol]

Il doit être loisible à tout-e herbonaute de marquer une Discussion comme étant « résolue », information qui peut être exploitée par le moteur de recherche des Messages et l'espace des questions non résolues (voir {\em supra} \in[disc:recherche] et \in[disc:espaceqnr]).

%%%%%%%%%%%%%%%%%%%%%%%%%%%%%%%%%%%%%%%%%%%%%%%%%%%%%%%%%%%%%%%%%%%%%%%%%%%%%%%%
%%%%%%%%%%%%%%%%%%%%%%%%%%%%%%%%%%%%%%%%%%%%%%%%%%%%%%%%%%%%%%%%%%%%%%%%%%%%%%%%
%%%%%%%%%%%%%%%%%%%%%%%%%%%%%%%%%%%%%%%%%%%%%%%%%%%%%%%%%%%%%%%%%%%%%%%%%%%%%%%%
\startsection[title={Moteur de recherche des Messages},reference=disc:recherche]

Le système devra proposer un moteur de recherche dans l'ensemble des Discussions produites.
En plus d'une recherche plein texte dans le corps des Messages et les titres des Discussions, celui-ci devra permettre de grouper les Messages selon leur Catégorie (\in[disc:categorie]) et de filtrer les Discussions suivant leur statut résolue/non résolue (\in[disc:resol]).

%%%%%%%%%%%%%%%%%%%%%%%%%%%%%%%%%%%%%%%%%%%%%%%%%%%%%%%%%%%%%%%%%%%%%%%%%%%%%%%%
%%%%%%%%%%%%%%%%%%%%%%%%%%%%%%%%%%%%%%%%%%%%%%%%%%%%%%%%%%%%%%%%%%%%%%%%%%%%%%%%
%%%%%%%%%%%%%%%%%%%%%%%%%%%%%%%%%%%%%%%%%%%%%%%%%%%%%%%%%%%%%%%%%%%%%%%%%%%%%%%%
\startsection[title={L'espace des Discussion}]

Le système doit proposer un espace agrégeant la totalité des Discussions produites dans le système, afin de constituer une base de connaissances tirant partie des Catégories (\in[disc:categorie]) et du moteur de recherche (\in[disc:recherche]).