\startchapter[title={Le modèle de discussions}]

%%%%%%%%%%%%%%%%%%%%%%%%%%%%%%%%%%%%%%%%%%%%%%%%%%%%%%%%%%%%%%%%%%%%%%%%%%%%%%%%
%%%%%%%%%%%%%%%%%%%%%%%%%%%%%%%%%%%%%%%%%%%%%%%%%%%%%%%%%%%%%%%%%%%%%%%%%%%%%%%%
%%%%%%%%%%%%%%%%%%%%%%%%%%%%%%%%%%%%%%%%%%%%%%%%%%%%%%%%%%%%%%%%%%%%%%%%%%%%%%%%
\startsection[title={Discussions ciblées & discussions générales}]

Nous proposons d'opérer une distinction entre, d'une part, les {\em discussions générales} abordant des points relatifs à la vie de la communauté, portant sur le fonctionnement d'un module, et d'autre part, les discussions de portée scientifique à propos d'un objet documentaire ou d'un concept scientifique identifié.
À la lumière de cette distinction, nous préconisons que les discussions générales se tiennent dans des composants de type forums classiques (nul besoin d'un développement spécifique pour ces fonctions standards d'animation de communauté).
À l'inverse, les discussions portant sur un objet botanique précis s'apparentent à des formes d'annotations dialogiques dont la valeur scientifique est indiscutable.
Le modèle proposé {\it infra} répond à cette exigence.

\stopsection
%%%%%%%%%%%%%%%%%%%%%%%%%%%%%%%%%%%%%%%%%%%%%%%%%%%%%%%%%%%%%%%%%%%%%%%%%%%%%%%%
%%%%%%%%%%%%%%%%%%%%%%%%%%%%%%%%%%%%%%%%%%%%%%%%%%%%%%%%%%%%%%%%%%%%%%%%%%%%%%%%
%%%%%%%%%%%%%%%%%%%%%%%%%%%%%%%%%%%%%%%%%%%%%%%%%%%%%%%%%%%%%%%%%%%%%%%%%%%%%%%%
\startsection[title={Le modèle de discussions},reference=model:discussions]

\fig{Discussions}{Diagramme de classes UML partiel du modèle de discussions}{f:uml:entities}{125}

Dans les termes du modèle e-ReColNat+, une discussion est un cas particulier d'ECAR, ce qui lui permet d'agréger comme ses sous-partie un ensemble de réponses, qui sont des EFARs.
Étudions les possibilités ouvertes par cette modélisation :

\startitemize
\item En tant qu'EAR, les discussions comme les réponses possèdent un/e utilisateur/trice créateur/trice, une date de création, et gardent trace du module ayant accueilli leur création. Ce dernier point permet non seulement d'amorcer une discussion portant sur un objet botanique dans un module et de la poursuivre dans un autre, mais, sur chaque écran de quelque module que ce soit représentant cette discussion, chaque réponse peut indiquer le module dans lequel elle a été formulée.
\item En tant qu'EAR et par la relation {\tt isAbout}, une discussion peut porter (et dans les faits, porte effectivement) sur une autre EAR (par exemple, une entité botanique).
\item Les discussions étant rattachées à des types concrets métier (planche, spécimen, étiquette, récolteur/trice, mission, etc.), le modèle rend possible des opérations de recherche de discussions par type d'entité, et donc des écrans dans les applications clientes de type « Les dernières discussions portant sur les planches ».
\stopitemize

\stopsection
\stopchapter