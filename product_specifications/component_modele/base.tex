\startchapter[title={Les entités de base du modèle e-RecolNat+},reference=model:base]

%%%%%%%%%%%%%%%%%%%%%%%%%%%%%%%%%%%%%%%%%%%%%%%%%%%%%%%%%%%%%%%%%%%%%%%%%%%%%%%%
%%%%%%%%%%%%%%%%%%%%%%%%%%%%%%%%%%%%%%%%%%%%%%%%%%%%%%%%%%%%%%%%%%%%%%%%%%%%%%%%
%%%%%%%%%%%%%%%%%%%%%%%%%%%%%%%%%%%%%%%%%%%%%%%%%%%%%%%%%%%%%%%%%%%%%%%%%%%%%%%%
\startsection[title={Introduction}]

Ce chapitre présente les fondements et les entités de base du modèle conceptuel e-ReColNat augmenté afin de préciser dans quel cadre logique nous pensons les rapports entre les modules dont le développement nous incombe et les autres modules du portail coopératif.
Les informations sur le modèle exposées dans ce document sont donc partielles, et insistent sur l'esprit d'ensemble plus que sur le détail des diverses classes de sous entités métier.

Le modèle e-RecolNat+ repose sur un petit ensemble d'entités abstraites destinées à être déclinées en entités métier et auxquelles elles fournissent des possibilités d'annotation et de structuration socialement et temporellement situées et scientifiquement qualifiées.
Sur la figure \in[f:uml:entities], ces entités sont marquées comme issues du groupe {\tt Base}.
Les concepts présentés ici, qui sont très abstraits, seront exemplifiés en \in[model:tags] et \in[model:discussions].

\fig{Entities}{Diagramme de classes UML simplifié du modèle de base e-RecolNat+}{f:uml:entities}{max}

\stopsection
%%%%%%%%%%%%%%%%%%%%%%%%%%%%%%%%%%%%%%%%%%%%%%%%%%%%%%%%%%%%%%%%%%%%%%%%%%%%%%%%
%%%%%%%%%%%%%%%%%%%%%%%%%%%%%%%%%%%%%%%%%%%%%%%%%%%%%%%%%%%%%%%%%%%%%%%%%%%%%%%%
%%%%%%%%%%%%%%%%%%%%%%%%%%%%%%%%%%%%%%%%%%%%%%%%%%%%%%%%%%%%%%%%%%%%%%%%%%%%%%%%
\startsection[title={Entité abstraite Recolnat (EAR)}]

Le concept d'entité Recolnat abstraite fournit le type fondamental du modèle e-ReColNat+ dont découle tout objet d'intérêt (à l'exception des entités scientifiques abstraites, voir \in[model:base:esa]).
Ce type se définit par les caractéristiques intrinsèques et relationnelles suivantes (voir figure \in[f:uml:entities]) :

\startitemize
\item Une EAR possède un identifiant e-ReColNat+ qui lui confère une identité stable au sein du système indépendamment de son type métier concret.
\item La création d'une EAR est située : tout objet sait à quel date, par qui (c'est-à-dire par quel utilisateur e-ReColNat) et dans quel module ou lieu du portail (collaboratoire, visite virtuelle, réseau social, Herbonautes, etc.) il a été créé.
\item Une EAR peut être associée à une entité scientifique abstraite (voir \in[model:base:esa]), et donc se rapporter à quelque chose qui existe en dehors de son module d'origine voire en dehors du système d'information e-ReColNat.
\item Une EAR peut se rapporter à une ou plusieurs EARs cibles (voir lien réflexif {\tt isAbout/relatedEntities} sur le diagramme de la figure \in[f:uml:entities]).
Ce lien dénote le positionnement fondamentalement annotationnel/critique du modèle e-ReColNat+, tout objet pouvant être relié à un autre par une relation hyperdocumentaire située et qualifiée (voir \in[model:base:situation]).
\stopitemize

\stopsection
%%%%%%%%%%%%%%%%%%%%%%%%%%%%%%%%%%%%%%%%%%%%%%%%%%%%%%%%%%%%%%%%%%%%%%%%%%%%%%%%
%%%%%%%%%%%%%%%%%%%%%%%%%%%%%%%%%%%%%%%%%%%%%%%%%%%%%%%%%%%%%%%%%%%%%%%%%%%%%%%%
%%%%%%%%%%%%%%%%%%%%%%%%%%%%%%%%%%%%%%%%%%%%%%%%%%%%%%%%%%%%%%%%%%%%%%%%%%%%%%%%
\startsection[title={Entité scientifique abstraite (ESA)},reference=model:base:esa]

Une entité scientifique abstraite représente un objet existant dans une base de données extérieure (base agrégée du WP2, ou toute autre base scientifique).
L'identité d'une ESA se compose de deux propriétés : un {\tt type} renvoyant au type de l'objet dans sa base de données scientifique de provenance, et un identifiant {\tt id} à valeur dans cette base\footnote{Il relève de la responsabilité des vues de construire des URLs d'accès effectif aux objets scientifique externes sur la base de ces deux champs.}.
Tout objet extérieur à la base e-ReColNat+ peut ainsi être désigné de manière univoque.

\stopsection
%%%%%%%%%%%%%%%%%%%%%%%%%%%%%%%%%%%%%%%%%%%%%%%%%%%%%%%%%%%%%%%%%%%%%%%%%%%%%%%%
%%%%%%%%%%%%%%%%%%%%%%%%%%%%%%%%%%%%%%%%%%%%%%%%%%%%%%%%%%%%%%%%%%%%%%%%%%%%%%%%
%%%%%%%%%%%%%%%%%%%%%%%%%%%%%%%%%%%%%%%%%%%%%%%%%%%%%%%%%%%%%%%%%%%%%%%%%%%%%%%%
\startsection[title={Situation},reference=model:base:situation]

La classe d'association {\tt Situation} telle qu'elle apparaît sur le diagramme de la figure \in[f:uml:entities] fournit un contexte à la relation réflexive {\tt isAbout} des EARs susmentionnée.
Chaque instance de cette relation est ainsi enrichie en premier lieu d'un type (propriété {\tt relationType}), dénotant la sémantique du lien documentaire ou critique qu'elle établit :

\startitemize
\item lien d'{\em enrichissement} (exemple : une annotation enrichit l'EAR sur laquelle elle porte) ;
\item lien d'{\em approbation} ou à l'inverse de {\em désaccord} (voir {\it infra}) ;
\item lien de {\em qualification} (exemple : un tag qualifie une EAR, voir \in[model:tags]) ;
\item lien d'{\em association} (exemple : un extrait de planche délimité dans le collaboratoire peut être associé à des ressources complémentaires telles que des notes ou des photos, ce qui constitue le fondement d'un carnet botanique hyperdocumentaire et hypermédia) ;
\item etc. (il est possible d'étendre le modèle avec d'autres sémantiques de lien réalisant des opérations critiques non prévues initialement).
\stopitemize
 
Ceci permet de typer les relations entre objets sans sacrifier l'unicité du graphe hyperdocumentaire auquel ils prennent part.
De plus, une Situation est également une EAR (au sens d'une relation UML de généralisation), ce qui confère à sa création la possibilité d'être datée et signée (par un utilisateur identifié).

Par ailleurs, en temps qu'EAR, une Situation jouit d'une identité propre qui lui permet à son tour d'être prise pour objet par la relation {\tt isAbout}.
Cette capacité peut être utilisée pour représenter le souhait d'un utilisateur de qualifier l'action d'un autre utilisateur portant sur la mise en relation de deux EARs.
Ceci appelle un exemple.
Soit un utilisateur U1 --- un aimable contributeur --- qui propose une transcription T1 pour un champ C1 de l'étiquette E1 d'une planche P1.
La réalisation de cette action générera une Situation S1 situant cette action dans le temps (date de création), dans l'espace social (l'auteur U1) et dans l'espace documentaire (mise en relation de T1 à C1 via un lien de type « enrichissement »\footnote{Une sémantique plus précise peut être admise, comme par exemple « Transcription », étant donné qu'il s'agit là d'un concept métier omniprésent.}).
La Situation S1 jouissant d'une identité propre, il devient possible à un autre utilisateur U2 --- un éminent botaniste reconnu comme tel par le composant d'authentification/habilitation --- de donner son avis sur celle-ci, par l'intermédiaire d'un lien de type « Approbation ».
La contribution T1 d'U1 est ainsi marquée comme contribution de premier plan (car reconnue par un chercheur « habilité »), ce qui peut constituer un critère de sélection intéressant lorsqu'il est question de compléter une base de données scientifique à partir des données contributives d'e-ReColNat+.

Plus généralement, le couplage des Situations et des liens typés inter-EARs permet de constituer un réseau critique complexe.

\stopsection
%%%%%%%%%%%%%%%%%%%%%%%%%%%%%%%%%%%%%%%%%%%%%%%%%%%%%%%%%%%%%%%%%%%%%%%%%%%%%%%%
%%%%%%%%%%%%%%%%%%%%%%%%%%%%%%%%%%%%%%%%%%%%%%%%%%%%%%%%%%%%%%%%%%%%%%%%%%%%%%%%
%%%%%%%%%%%%%%%%%%%%%%%%%%%%%%%%%%%%%%%%%%%%%%%%%%%%%%%%%%%%%%%%%%%%%%%%%%%%%%%%
\startsection[title={Sous-types d'EARs}]

Outre les Situations, qui ont un statut particulier du fait de leur fonction d'opérateurs de contextualisation de l'ensemble des opérations de mise en relation permises par le système, les EARs se déclinent en deux sous-types principaux : les entités feuilles abstraites Recolnat et les entités composites abstraites Recolnat.
Ces entités, présentées {\it infra}, renvoient au \goto{patron de conception « composite »}[url(http://en.wikipedia.org/wiki/Composite_pattern)].

%%%%%%%%%%%%%%%%%%%%%%%%%%%%%%%%%%%%%%%%%%%%%%%%%%%%%%%%%%%%%%%%%%%%%%%%%%%%%%%%
\startsubsection[title={Entité feuille abstraite Recolnat (EFAR)}]

Une entité feuille abstraite Recolnat représente une information monadique.
Le diagramme de la figure \in[f:uml:entities] en donne les exemples suivants : commentaire simple, transcription, détermination, nom vernaculaire, coordonnées géographiques.
Dans tous ces cas, le contenu de l'EFAR est indivisible, n'admet aucune sous-partie dotée d'une identité propre et jouissant d'un certain degré d'autonomie.

\stopsubsection
%%%%%%%%%%%%%%%%%%%%%%%%%%%%%%%%%%%%%%%%%%%%%%%%%%%%%%%%%%%%%%%%%%%%%%%%%%%%%%%%
\startsubsection[title={Entité composite abstraite Recolnat (ECAR)}]

Une entité composite abstraite Recolnat admet un ensemble d'EAR définies comme ses sous-parties.
Les types concrets métier du diagramme de la figure \in[f:uml:entities] héritant d'ECAR illustrent tous cette propriété structurelle : un herbier {\it contient} des planches, une planche {\it se compose de} sous-parties d'intérêt (étiquette, feuille identifiée, etc.), un/e conservateur/trice {\it a la responsabilité} d'un ou plusieurs herbiers, une récolte {\it contient} des spécimens, une collection {\it contient} des herbiers/planches, etc.
Par ailleurs, à l'instar de la relation {\tt isAbout}, le lien d'agrégation {\tt containedEntities} entre une ECAR et un ensemble d'EARs implique une EAR Situation en tant que classe d'association.
Ceci permet la contextualisation et la critique de la création des relations d'appartenance (voir \in[model:base:situation]).

\stopsubsection
\stopsection
\stopchapter